
% Default to the notebook output style

    


% Inherit from the specified cell style.




    
\documentclass[11pt]{article}

    
    
    \usepackage[T1]{fontenc}
    % Nicer default font (+ math font) than Computer Modern for most use cases
    \usepackage{mathpazo}

    % Basic figure setup, for now with no caption control since it's done
    % automatically by Pandoc (which extracts ![](path) syntax from Markdown).
    \usepackage{graphicx}
    % We will generate all images so they have a width \maxwidth. This means
    % that they will get their normal width if they fit onto the page, but
    % are scaled down if they would overflow the margins.
    \makeatletter
    \def\maxwidth{\ifdim\Gin@nat@width>\linewidth\linewidth
    \else\Gin@nat@width\fi}
    \makeatother
    \let\Oldincludegraphics\includegraphics
    % Set max figure width to be 80% of text width, for now hardcoded.
    \renewcommand{\includegraphics}[1]{\Oldincludegraphics[width=.8\maxwidth]{#1}}
    % Ensure that by default, figures have no caption (until we provide a
    % proper Figure object with a Caption API and a way to capture that
    % in the conversion process - todo).
    \usepackage{caption}
    \DeclareCaptionLabelFormat{nolabel}{}
    \captionsetup{labelformat=nolabel}

    \usepackage{adjustbox} % Used to constrain images to a maximum size 
    \usepackage{xcolor} % Allow colors to be defined
    \usepackage{enumerate} % Needed for markdown enumerations to work
    \usepackage{geometry} % Used to adjust the document margins
    \usepackage{amsmath} % Equations
    \usepackage{amssymb} % Equations
    \usepackage{textcomp} % defines textquotesingle
    % Hack from http://tex.stackexchange.com/a/47451/13684:
    \AtBeginDocument{%
        \def\PYZsq{\textquotesingle}% Upright quotes in Pygmentized code
    }
    \usepackage{upquote} % Upright quotes for verbatim code
    \usepackage{eurosym} % defines \euro
    \usepackage[mathletters]{ucs} % Extended unicode (utf-8) support
    \usepackage[utf8x]{inputenc} % Allow utf-8 characters in the tex document
    \usepackage{fancyvrb} % verbatim replacement that allows latex
    \usepackage{grffile} % extends the file name processing of package graphics 
                         % to support a larger range 
    % The hyperref package gives us a pdf with properly built
    % internal navigation ('pdf bookmarks' for the table of contents,
    % internal cross-reference links, web links for URLs, etc.)
    \usepackage{hyperref}
    \usepackage{longtable} % longtable support required by pandoc >1.10
    \usepackage{booktabs}  % table support for pandoc > 1.12.2
    \usepackage[inline]{enumitem} % IRkernel/repr support (it uses the enumerate* environment)
    \usepackage[normalem]{ulem} % ulem is needed to support strikethroughs (\sout)
                                % normalem makes italics be italics, not underlines
    

    
    
    % Colors for the hyperref package
    \definecolor{urlcolor}{rgb}{0,.145,.698}
    \definecolor{linkcolor}{rgb}{.71,0.21,0.01}
    \definecolor{citecolor}{rgb}{.12,.54,.11}

    % ANSI colors
    \definecolor{ansi-black}{HTML}{3E424D}
    \definecolor{ansi-black-intense}{HTML}{282C36}
    \definecolor{ansi-red}{HTML}{E75C58}
    \definecolor{ansi-red-intense}{HTML}{B22B31}
    \definecolor{ansi-green}{HTML}{00A250}
    \definecolor{ansi-green-intense}{HTML}{007427}
    \definecolor{ansi-yellow}{HTML}{DDB62B}
    \definecolor{ansi-yellow-intense}{HTML}{B27D12}
    \definecolor{ansi-blue}{HTML}{208FFB}
    \definecolor{ansi-blue-intense}{HTML}{0065CA}
    \definecolor{ansi-magenta}{HTML}{D160C4}
    \definecolor{ansi-magenta-intense}{HTML}{A03196}
    \definecolor{ansi-cyan}{HTML}{60C6C8}
    \definecolor{ansi-cyan-intense}{HTML}{258F8F}
    \definecolor{ansi-white}{HTML}{C5C1B4}
    \definecolor{ansi-white-intense}{HTML}{A1A6B2}

    % commands and environments needed by pandoc snippets
    % extracted from the output of `pandoc -s`
    \providecommand{\tightlist}{%
      \setlength{\itemsep}{0pt}\setlength{\parskip}{0pt}}
    \DefineVerbatimEnvironment{Highlighting}{Verbatim}{commandchars=\\\{\}}
    % Add ',fontsize=\small' for more characters per line
    \newenvironment{Shaded}{}{}
    \newcommand{\KeywordTok}[1]{\textcolor[rgb]{0.00,0.44,0.13}{\textbf{{#1}}}}
    \newcommand{\DataTypeTok}[1]{\textcolor[rgb]{0.56,0.13,0.00}{{#1}}}
    \newcommand{\DecValTok}[1]{\textcolor[rgb]{0.25,0.63,0.44}{{#1}}}
    \newcommand{\BaseNTok}[1]{\textcolor[rgb]{0.25,0.63,0.44}{{#1}}}
    \newcommand{\FloatTok}[1]{\textcolor[rgb]{0.25,0.63,0.44}{{#1}}}
    \newcommand{\CharTok}[1]{\textcolor[rgb]{0.25,0.44,0.63}{{#1}}}
    \newcommand{\StringTok}[1]{\textcolor[rgb]{0.25,0.44,0.63}{{#1}}}
    \newcommand{\CommentTok}[1]{\textcolor[rgb]{0.38,0.63,0.69}{\textit{{#1}}}}
    \newcommand{\OtherTok}[1]{\textcolor[rgb]{0.00,0.44,0.13}{{#1}}}
    \newcommand{\AlertTok}[1]{\textcolor[rgb]{1.00,0.00,0.00}{\textbf{{#1}}}}
    \newcommand{\FunctionTok}[1]{\textcolor[rgb]{0.02,0.16,0.49}{{#1}}}
    \newcommand{\RegionMarkerTok}[1]{{#1}}
    \newcommand{\ErrorTok}[1]{\textcolor[rgb]{1.00,0.00,0.00}{\textbf{{#1}}}}
    \newcommand{\NormalTok}[1]{{#1}}
    
    % Additional commands for more recent versions of Pandoc
    \newcommand{\ConstantTok}[1]{\textcolor[rgb]{0.53,0.00,0.00}{{#1}}}
    \newcommand{\SpecialCharTok}[1]{\textcolor[rgb]{0.25,0.44,0.63}{{#1}}}
    \newcommand{\VerbatimStringTok}[1]{\textcolor[rgb]{0.25,0.44,0.63}{{#1}}}
    \newcommand{\SpecialStringTok}[1]{\textcolor[rgb]{0.73,0.40,0.53}{{#1}}}
    \newcommand{\ImportTok}[1]{{#1}}
    \newcommand{\DocumentationTok}[1]{\textcolor[rgb]{0.73,0.13,0.13}{\textit{{#1}}}}
    \newcommand{\AnnotationTok}[1]{\textcolor[rgb]{0.38,0.63,0.69}{\textbf{\textit{{#1}}}}}
    \newcommand{\CommentVarTok}[1]{\textcolor[rgb]{0.38,0.63,0.69}{\textbf{\textit{{#1}}}}}
    \newcommand{\VariableTok}[1]{\textcolor[rgb]{0.10,0.09,0.49}{{#1}}}
    \newcommand{\ControlFlowTok}[1]{\textcolor[rgb]{0.00,0.44,0.13}{\textbf{{#1}}}}
    \newcommand{\OperatorTok}[1]{\textcolor[rgb]{0.40,0.40,0.40}{{#1}}}
    \newcommand{\BuiltInTok}[1]{{#1}}
    \newcommand{\ExtensionTok}[1]{{#1}}
    \newcommand{\PreprocessorTok}[1]{\textcolor[rgb]{0.74,0.48,0.00}{{#1}}}
    \newcommand{\AttributeTok}[1]{\textcolor[rgb]{0.49,0.56,0.16}{{#1}}}
    \newcommand{\InformationTok}[1]{\textcolor[rgb]{0.38,0.63,0.69}{\textbf{\textit{{#1}}}}}
    \newcommand{\WarningTok}[1]{\textcolor[rgb]{0.38,0.63,0.69}{\textbf{\textit{{#1}}}}}
    
    
    % Define a nice break command that doesn't care if a line doesn't already
    % exist.
    \def\br{\hspace*{\fill} \\* }
    % Math Jax compatability definitions
    \def\gt{>}
    \def\lt{<}
    % Document parameters
    \title{ENGR114-Syllabus-2019Q1}
    
    
    

    % Pygments definitions
    
\makeatletter
\def\PY@reset{\let\PY@it=\relax \let\PY@bf=\relax%
    \let\PY@ul=\relax \let\PY@tc=\relax%
    \let\PY@bc=\relax \let\PY@ff=\relax}
\def\PY@tok#1{\csname PY@tok@#1\endcsname}
\def\PY@toks#1+{\ifx\relax#1\empty\else%
    \PY@tok{#1}\expandafter\PY@toks\fi}
\def\PY@do#1{\PY@bc{\PY@tc{\PY@ul{%
    \PY@it{\PY@bf{\PY@ff{#1}}}}}}}
\def\PY#1#2{\PY@reset\PY@toks#1+\relax+\PY@do{#2}}

\expandafter\def\csname PY@tok@w\endcsname{\def\PY@tc##1{\textcolor[rgb]{0.73,0.73,0.73}{##1}}}
\expandafter\def\csname PY@tok@c\endcsname{\let\PY@it=\textit\def\PY@tc##1{\textcolor[rgb]{0.25,0.50,0.50}{##1}}}
\expandafter\def\csname PY@tok@cp\endcsname{\def\PY@tc##1{\textcolor[rgb]{0.74,0.48,0.00}{##1}}}
\expandafter\def\csname PY@tok@k\endcsname{\let\PY@bf=\textbf\def\PY@tc##1{\textcolor[rgb]{0.00,0.50,0.00}{##1}}}
\expandafter\def\csname PY@tok@kp\endcsname{\def\PY@tc##1{\textcolor[rgb]{0.00,0.50,0.00}{##1}}}
\expandafter\def\csname PY@tok@kt\endcsname{\def\PY@tc##1{\textcolor[rgb]{0.69,0.00,0.25}{##1}}}
\expandafter\def\csname PY@tok@o\endcsname{\def\PY@tc##1{\textcolor[rgb]{0.40,0.40,0.40}{##1}}}
\expandafter\def\csname PY@tok@ow\endcsname{\let\PY@bf=\textbf\def\PY@tc##1{\textcolor[rgb]{0.67,0.13,1.00}{##1}}}
\expandafter\def\csname PY@tok@nb\endcsname{\def\PY@tc##1{\textcolor[rgb]{0.00,0.50,0.00}{##1}}}
\expandafter\def\csname PY@tok@nf\endcsname{\def\PY@tc##1{\textcolor[rgb]{0.00,0.00,1.00}{##1}}}
\expandafter\def\csname PY@tok@nc\endcsname{\let\PY@bf=\textbf\def\PY@tc##1{\textcolor[rgb]{0.00,0.00,1.00}{##1}}}
\expandafter\def\csname PY@tok@nn\endcsname{\let\PY@bf=\textbf\def\PY@tc##1{\textcolor[rgb]{0.00,0.00,1.00}{##1}}}
\expandafter\def\csname PY@tok@ne\endcsname{\let\PY@bf=\textbf\def\PY@tc##1{\textcolor[rgb]{0.82,0.25,0.23}{##1}}}
\expandafter\def\csname PY@tok@nv\endcsname{\def\PY@tc##1{\textcolor[rgb]{0.10,0.09,0.49}{##1}}}
\expandafter\def\csname PY@tok@no\endcsname{\def\PY@tc##1{\textcolor[rgb]{0.53,0.00,0.00}{##1}}}
\expandafter\def\csname PY@tok@nl\endcsname{\def\PY@tc##1{\textcolor[rgb]{0.63,0.63,0.00}{##1}}}
\expandafter\def\csname PY@tok@ni\endcsname{\let\PY@bf=\textbf\def\PY@tc##1{\textcolor[rgb]{0.60,0.60,0.60}{##1}}}
\expandafter\def\csname PY@tok@na\endcsname{\def\PY@tc##1{\textcolor[rgb]{0.49,0.56,0.16}{##1}}}
\expandafter\def\csname PY@tok@nt\endcsname{\let\PY@bf=\textbf\def\PY@tc##1{\textcolor[rgb]{0.00,0.50,0.00}{##1}}}
\expandafter\def\csname PY@tok@nd\endcsname{\def\PY@tc##1{\textcolor[rgb]{0.67,0.13,1.00}{##1}}}
\expandafter\def\csname PY@tok@s\endcsname{\def\PY@tc##1{\textcolor[rgb]{0.73,0.13,0.13}{##1}}}
\expandafter\def\csname PY@tok@sd\endcsname{\let\PY@it=\textit\def\PY@tc##1{\textcolor[rgb]{0.73,0.13,0.13}{##1}}}
\expandafter\def\csname PY@tok@si\endcsname{\let\PY@bf=\textbf\def\PY@tc##1{\textcolor[rgb]{0.73,0.40,0.53}{##1}}}
\expandafter\def\csname PY@tok@se\endcsname{\let\PY@bf=\textbf\def\PY@tc##1{\textcolor[rgb]{0.73,0.40,0.13}{##1}}}
\expandafter\def\csname PY@tok@sr\endcsname{\def\PY@tc##1{\textcolor[rgb]{0.73,0.40,0.53}{##1}}}
\expandafter\def\csname PY@tok@ss\endcsname{\def\PY@tc##1{\textcolor[rgb]{0.10,0.09,0.49}{##1}}}
\expandafter\def\csname PY@tok@sx\endcsname{\def\PY@tc##1{\textcolor[rgb]{0.00,0.50,0.00}{##1}}}
\expandafter\def\csname PY@tok@m\endcsname{\def\PY@tc##1{\textcolor[rgb]{0.40,0.40,0.40}{##1}}}
\expandafter\def\csname PY@tok@gh\endcsname{\let\PY@bf=\textbf\def\PY@tc##1{\textcolor[rgb]{0.00,0.00,0.50}{##1}}}
\expandafter\def\csname PY@tok@gu\endcsname{\let\PY@bf=\textbf\def\PY@tc##1{\textcolor[rgb]{0.50,0.00,0.50}{##1}}}
\expandafter\def\csname PY@tok@gd\endcsname{\def\PY@tc##1{\textcolor[rgb]{0.63,0.00,0.00}{##1}}}
\expandafter\def\csname PY@tok@gi\endcsname{\def\PY@tc##1{\textcolor[rgb]{0.00,0.63,0.00}{##1}}}
\expandafter\def\csname PY@tok@gr\endcsname{\def\PY@tc##1{\textcolor[rgb]{1.00,0.00,0.00}{##1}}}
\expandafter\def\csname PY@tok@ge\endcsname{\let\PY@it=\textit}
\expandafter\def\csname PY@tok@gs\endcsname{\let\PY@bf=\textbf}
\expandafter\def\csname PY@tok@gp\endcsname{\let\PY@bf=\textbf\def\PY@tc##1{\textcolor[rgb]{0.00,0.00,0.50}{##1}}}
\expandafter\def\csname PY@tok@go\endcsname{\def\PY@tc##1{\textcolor[rgb]{0.53,0.53,0.53}{##1}}}
\expandafter\def\csname PY@tok@gt\endcsname{\def\PY@tc##1{\textcolor[rgb]{0.00,0.27,0.87}{##1}}}
\expandafter\def\csname PY@tok@err\endcsname{\def\PY@bc##1{\setlength{\fboxsep}{0pt}\fcolorbox[rgb]{1.00,0.00,0.00}{1,1,1}{\strut ##1}}}
\expandafter\def\csname PY@tok@kc\endcsname{\let\PY@bf=\textbf\def\PY@tc##1{\textcolor[rgb]{0.00,0.50,0.00}{##1}}}
\expandafter\def\csname PY@tok@kd\endcsname{\let\PY@bf=\textbf\def\PY@tc##1{\textcolor[rgb]{0.00,0.50,0.00}{##1}}}
\expandafter\def\csname PY@tok@kn\endcsname{\let\PY@bf=\textbf\def\PY@tc##1{\textcolor[rgb]{0.00,0.50,0.00}{##1}}}
\expandafter\def\csname PY@tok@kr\endcsname{\let\PY@bf=\textbf\def\PY@tc##1{\textcolor[rgb]{0.00,0.50,0.00}{##1}}}
\expandafter\def\csname PY@tok@bp\endcsname{\def\PY@tc##1{\textcolor[rgb]{0.00,0.50,0.00}{##1}}}
\expandafter\def\csname PY@tok@fm\endcsname{\def\PY@tc##1{\textcolor[rgb]{0.00,0.00,1.00}{##1}}}
\expandafter\def\csname PY@tok@vc\endcsname{\def\PY@tc##1{\textcolor[rgb]{0.10,0.09,0.49}{##1}}}
\expandafter\def\csname PY@tok@vg\endcsname{\def\PY@tc##1{\textcolor[rgb]{0.10,0.09,0.49}{##1}}}
\expandafter\def\csname PY@tok@vi\endcsname{\def\PY@tc##1{\textcolor[rgb]{0.10,0.09,0.49}{##1}}}
\expandafter\def\csname PY@tok@vm\endcsname{\def\PY@tc##1{\textcolor[rgb]{0.10,0.09,0.49}{##1}}}
\expandafter\def\csname PY@tok@sa\endcsname{\def\PY@tc##1{\textcolor[rgb]{0.73,0.13,0.13}{##1}}}
\expandafter\def\csname PY@tok@sb\endcsname{\def\PY@tc##1{\textcolor[rgb]{0.73,0.13,0.13}{##1}}}
\expandafter\def\csname PY@tok@sc\endcsname{\def\PY@tc##1{\textcolor[rgb]{0.73,0.13,0.13}{##1}}}
\expandafter\def\csname PY@tok@dl\endcsname{\def\PY@tc##1{\textcolor[rgb]{0.73,0.13,0.13}{##1}}}
\expandafter\def\csname PY@tok@s2\endcsname{\def\PY@tc##1{\textcolor[rgb]{0.73,0.13,0.13}{##1}}}
\expandafter\def\csname PY@tok@sh\endcsname{\def\PY@tc##1{\textcolor[rgb]{0.73,0.13,0.13}{##1}}}
\expandafter\def\csname PY@tok@s1\endcsname{\def\PY@tc##1{\textcolor[rgb]{0.73,0.13,0.13}{##1}}}
\expandafter\def\csname PY@tok@mb\endcsname{\def\PY@tc##1{\textcolor[rgb]{0.40,0.40,0.40}{##1}}}
\expandafter\def\csname PY@tok@mf\endcsname{\def\PY@tc##1{\textcolor[rgb]{0.40,0.40,0.40}{##1}}}
\expandafter\def\csname PY@tok@mh\endcsname{\def\PY@tc##1{\textcolor[rgb]{0.40,0.40,0.40}{##1}}}
\expandafter\def\csname PY@tok@mi\endcsname{\def\PY@tc##1{\textcolor[rgb]{0.40,0.40,0.40}{##1}}}
\expandafter\def\csname PY@tok@il\endcsname{\def\PY@tc##1{\textcolor[rgb]{0.40,0.40,0.40}{##1}}}
\expandafter\def\csname PY@tok@mo\endcsname{\def\PY@tc##1{\textcolor[rgb]{0.40,0.40,0.40}{##1}}}
\expandafter\def\csname PY@tok@ch\endcsname{\let\PY@it=\textit\def\PY@tc##1{\textcolor[rgb]{0.25,0.50,0.50}{##1}}}
\expandafter\def\csname PY@tok@cm\endcsname{\let\PY@it=\textit\def\PY@tc##1{\textcolor[rgb]{0.25,0.50,0.50}{##1}}}
\expandafter\def\csname PY@tok@cpf\endcsname{\let\PY@it=\textit\def\PY@tc##1{\textcolor[rgb]{0.25,0.50,0.50}{##1}}}
\expandafter\def\csname PY@tok@c1\endcsname{\let\PY@it=\textit\def\PY@tc##1{\textcolor[rgb]{0.25,0.50,0.50}{##1}}}
\expandafter\def\csname PY@tok@cs\endcsname{\let\PY@it=\textit\def\PY@tc##1{\textcolor[rgb]{0.25,0.50,0.50}{##1}}}

\def\PYZbs{\char`\\}
\def\PYZus{\char`\_}
\def\PYZob{\char`\{}
\def\PYZcb{\char`\}}
\def\PYZca{\char`\^}
\def\PYZam{\char`\&}
\def\PYZlt{\char`\<}
\def\PYZgt{\char`\>}
\def\PYZsh{\char`\#}
\def\PYZpc{\char`\%}
\def\PYZdl{\char`\$}
\def\PYZhy{\char`\-}
\def\PYZsq{\char`\'}
\def\PYZdq{\char`\"}
\def\PYZti{\char`\~}
% for compatibility with earlier versions
\def\PYZat{@}
\def\PYZlb{[}
\def\PYZrb{]}
\makeatother


    % Exact colors from NB
    \definecolor{incolor}{rgb}{0.0, 0.0, 0.5}
    \definecolor{outcolor}{rgb}{0.545, 0.0, 0.0}



    
    % Prevent overflowing lines due to hard-to-break entities
    \sloppy 
    % Setup hyperref package
    \hypersetup{
      breaklinks=true,  % so long urls are correctly broken across lines
      colorlinks=true,
      urlcolor=urlcolor,
      linkcolor=linkcolor,
      citecolor=citecolor,
      }
    % Slightly bigger margins than the latex defaults
    
    \geometry{verbose,tmargin=1in,bmargin=1in,lmargin=1in,rmargin=1in}
    
    

    \begin{document}
    
    
    \maketitle
    
    

    
    \section{ENGR114 Syllabus - Winter
2019}\label{engr114-syllabus---winter-2019}

    \subsection{Instructor Information}\label{instructor-information}

\begin{itemize}
\tightlist
\item
  Instructor:
\item
  Instructor email:
\item
  Instructor phone: (email is the best way to get in touch)
\item
  Office Hours:
\item
  Office Location: SY ST200
\end{itemize}

    \subsection{Course Information}\label{course-information}

\subsubsection{Course Description}\label{course-description}

\begin{itemize}
\tightlist
\item
  credits:
\item
  Prerequisites:
\end{itemize}

\subsubsection{Course Meeting Times}\label{course-meeting-times}

\begin{itemize}
\tightlist
\item
  Lecture:
\item
  Lab:
\end{itemize}

\subsection{Course Materials}\label{course-materials}

\subsubsection{Textbook}\label{textbook}

\emph{Problem Solving with Python} by Peter D. Kazarinoff. Available on
LeanPub.com and online at
\href{https://problemsovlingwithpython.com}{problemsolvingwithpython.com}

\subsubsection{Other course materials}\label{other-course-materials}

You need access the course programming environment, available at
\href{https://engr114.org\%5D}{https://engr114.org}. An @pcc.edu email
address and password is needed to login.

All PCC Computer Labs have Internet access. You need to be signed into
your myPCC email account before you log in to the course programming
environment. The course programming environment will work on all student
laptops, desktops, and chromebooks. But remember, you need an Internet
connection to use it.

    \subsection{Course Components}\label{course-components}

    \subsubsection{Homework}\label{homework}

Homework is intended as a means of practicing the concepts presented in
class. Timely submission of homework via Desire2Learn (D2L) is required.
\textbf{No late homework is accepted in this course} The D2L upload
folder will close to submissions after the assignment deadline. With
this in mind, your \textbf{lowest homework score will be dropped}. All
homework assignments are weighted equally. Check D2L for assignments,
homework format and submission deadlines.

    \subsubsection{Exams}\label{exams}

It is the student's responsibility to notify the instructor \textbf{at
least 3 days prior to the date of an exam} if the student is unable to
take the exam when it is scheduled. Make-up exams, if allowed, will be
scheduled by the instructor after consulting with the student
\textbf{prior} to the original date of the exam. I look forward to
working with any students that have testing accommodations. It is the
student's responsibility to set up a meeting \textbf{at least 1 week
prior to the exam date} to review exam protocol as the testing center
does not have laptops with the course programming environment installed.

\paragraph{During Exams:}\label{during-exams}

\begin{itemize}
\tightlist
\item
  Exams will be taken during lecture or lab time in the classroom.
\item
  Exams will have two parts, a written portion and a computer portion
\item
  During the written portion you can not use Python and computer
  monitors must be off
\item
  During the computer portion you must use Python on the computers in
  the classroom. No student laptops allowed on exams.
\item
  Generally, a calculator and one page of notes are allowed on the
  exams.
\item
  You must complete the Exam during the exam period. Uploads after the
  deadline will not receive credit.
\end{itemize}

    \subsubsection{Labs}\label{labs}

Typically, lecture will be held early in the week and lab held later in
the week. This is subject to change. Lab assignments are typically due
at the end of the lab period, uploaded to D2L. \textbf{No late labs will
be accepted}. \textbf{You must be present in lab to receive credit for
the lab assignment}. Your \textbf{lowest lab score will be dropped}. Lab
assignments and grading are subject to change throughout the quarter.

    \subsubsection{In-class assignments}\label{in-class-assignments}

Most lecture days will include an in-class assignment that needs to be
uploaded to D2L by the end of the class period. See D2L for specific
in-class assignment details. \textbf{No late in-class assignments will
be accepted}. \textbf{You must be present in lecture to receive credit
for the in-class assignment}. Your \textbf{lowest in-class assignment
score will be dropped}. In-class assignments and grading are subject to
change throughout the quarter.

    \subsubsection{Project}\label{project}

You will complete a group project over the course of the quarter. This
project is a group assignment. Project delivery will be at the end of
the quarter. Students must be present when their project is delivered to
receive credit. During the quarter there will be occasional project
assignments to complete. See D2L for the project description and grading
details. \textbf{No late projects will be accepted}.

    \subsection{Grading}\label{grading}

\begin{itemize}
\tightlist
\item
  Homework: 15\%
\item
  Exams: 20\%
\item
  Labs: 25\%
\item
  In-class: 5\%
\item
  Project: 15\%
\end{itemize}

Grades will be rounded to the nearest two significant figures (nearest
whole percentage point) and converted to letter grades according to the
following. 84.4999999\% will round to an 84\%, 84.5000\% will round to
85\%.

100-90 = A

80 - 89 = B

70 - 79 = C

65 - 69 = D

0 - 64 = F

    \subsection{Class rules and resources}\label{class-rules-and-resources}

\subsubsection{Class Rules}\label{class-rules}

Be respectful and courteous to others. Failure to do so may result in
the lowering of the course grade. Cheating will not be tolerated, and
may result in failure of the entire course.

\subsubsection{Attendance, Make-Up, and Other Class
Policies}\label{attendance-make-up-and-other-class-policies}

Attendance at all labs and lectures is required. Attendance may be used
in the grading process. Tardy students may be denied entry to class or
docked in credit for in-class assignments. Poor attendance or repeated
tardiness will have a negative effect on a student's grade for the
course.

Students should check their PCC email at least three times a week to
ensure receipt of announcements and important information about the
class.

Students may only attend this course if registered. Students who are
unable to attend must drop the course online or through the Registration
Office. To have tuition charges removed, the course must be dropped by
the student before the drop deadline posted on MyPCC. Students who never
attend, or stop attending, without dropping may receive a W or a failing
grade and will be required to pay for the course.

The instructor reserves the right to modify course content and/or
substitute assignments and learning activities in response to
institutional, weather or class situations. To find out if the college
is closed due to bad weather please check the
\href{http://www.pcc.edu/}{PCC website}, call the PCC switchboard at
503-244-6111, check the local radio and TV stations, or subscribe to
\href{https://www.pcc.edu/about/announcements/closure-information.html}{FlashAlerts}.

    \subsubsection{Additional Content Areas, Links, and Other
Information}\label{additional-content-areas-links-and-other-information}

(Note: See electronic copy for clickable links to these resources.)

\begin{itemize}
\item
  \href{http://greenteapress.com/think-python}{Think Python} is a free
  textbook in PDF form, or is available for purchase in print form from
  Amazon.
\item
  \href{http://www.pcc.edu/resources/academic/standards-practices/AcademicStandardsandPractices-GradingGuidelines.html}{PCC
  Grading Guidelines}
\item
  \href{http://www.pcc.edu/registration/dropping.html}{Add/drop/withdraw
  deadlines for the term}
\item
  \href{http://www.pcc.edu/registration/academic-calendar.html}{PCC
  Academic Calendar}
\item
  \href{https://www.pcc.edu/about/equity-inclusion/eeo-statement.html}{PCC
  Nondiscrimination Statement}
\end{itemize}

\paragraph{Student services:}\label{student-services}

\href{https://www.pcc.edu/resources/veterans/}{Veteran Resources},

\href{http://www.pcc.edu/resources/tutoring/sylvania/student-success/}{Sylvania Student Learning Center},

\href{https://www.pcc.edu/resources/women/sylvania/}{Sylvania Women's Resource Center},
* *

\{https://www.pcc.edu/resources/qrc/sylvania/\}\{Queer Resources and
Programming at Sylvania\}

\href{http://www.pcc.edu/about/policy/student-rights/}{Student Rights and Responsibilities},
Includes information on Children on PCC Properties, Academic Integrity
Policy, Policy on Student Conduct

\begin{itemize}
\tightlist
\item
  Use of cell phones and other personal electronic devices, unless
  otherwise stated, is not allowed during classes, labs, or exams.
\item
  The unauthorized recording (audio, video, or still pictures) of course
  sessions and distribution of said recordings is not permitted.
\item
  The instructor reserves the right to modify course content and/or
  substitute assignments and learning activities in response to
  institutional, weather, or class situations.
\end{itemize}

\textbackslash{}end\{itemize\}

    \subsubsection{Students with
disabilities}\label{students-with-disabilities}

Students who have a documented disability and require accommodation
should contact
\href{http://www.pcc.edu/resources/disability}{Disabilities Services}
and provide the Approved Academic Accommodations letter to the
Instructor.

    \subsubsection{Title IV}\label{title-iv}

Portland Community College is committed to creating and fostering a
learning and working environment based on open communication and mutual
respect. If you believe you have encountered sexual harassment, sexual
misconduct, sexual assault, or discrimination based on race, color,
religion, age, national origin, veteran status, sex, sexual orientation,
gender identity, or disability please contact the Office of Equity and
Inclusion at (971) 722-5840 or equity.inclusion@pcc.edu.

    \subsubsection{Academic Integrity}\label{academic-integrity}

Students are required to complete this course in accordance with the PCC
Student Rights and Responsibilities Handbook. Dishonest activities such
as cheating on exams and submitting or copying work done by others will
result in disciplinary actions including, but not limited to, receiving
a failing grade in the course.

    \subsection{Tentative Course Schedule}\label{tentative-course-schedule}

The course schedule below is subject to change. Check D2L for specific
course deadlines.

\begin{longtable}[]{@{}lll@{}}
\toprule
Week & Lecture & Lab\tabularnewline
\midrule
\endhead
Week 1 & The Python REPL & Lab 1: Calculations\tabularnewline
Week 2 & Data types and variables & Lab 2: Data Types\tabularnewline
Week 3 & Data structures, input and print & Lab 3: User
Input\tabularnewline
Week 4 & Plotting & Lab 4: Plotting\tabularnewline
Week 5 & Functions & Lab 5: Web API's\tabularnewline
Week 6 & Review & \textbf{Exam 1}\tabularnewline
Week 7 & If, elif, else & Lab 6: Sound filters\tabularnewline
Week 8 & Loops & Lab 7: Taylor Series\tabularnewline
Week 9 & MicroPython & Lab 7: MicroPython\tabularnewline
Week 9 & External Hardware & Lab 7: External Hardware\tabularnewline
Week 10 & Review & \textbf{Exam 2}\tabularnewline
Finals Week & Project Delivery &\tabularnewline
\bottomrule
\end{longtable}


    % Add a bibliography block to the postdoc
    
    
    
    \end{document}

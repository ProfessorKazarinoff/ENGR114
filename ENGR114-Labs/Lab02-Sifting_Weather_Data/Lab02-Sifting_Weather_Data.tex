%% ENGR114_lab_assignment.tplx %%
%
% Built off of the article.tplx template %


% Default to the notebook output style

    


% Inherit from the specified cell style.




    
    \documentclass[11pt]{article}

    
    
    %% installed packages_rev2.tplx %%

\usepackage{fancyhdr}
\usepackage{lastpage}
\usepackage{framed,color}
\definecolor{shadecolor}{rgb}{.8,.8,.8}
\usepackage{titlesec}
% no indent on any paragraphs, vertical spacing between paragraphs is set to 1em
\usepackage[]{parskip}  % add [skip=1em] if the compiler will allow.

% for MATLAB syntax highlighting
\usepackage{listings}             % Include the listings-package
\definecolor{mygray}{rgb}{0.8,0.8,0.8} % color values Red, Green, Blue
\definecolor{mygreen}{RGB}{28,172,0}
\definecolor{mylilas}{RGB}{170,55,241}
    
    \usepackage[T1]{fontenc}
    % Nicer default font (+ math font) than Computer Modern for most use cases
    \usepackage{mathpazo}

    % Basic figure setup, for now with no caption control since it's done
    % automatically by Pandoc (which extracts ![](path) syntax from Markdown).
    \usepackage{graphicx}
    % We will generate all images so they have a width \maxwidth. This means
    % that they will get their normal width if they fit onto the page, but
    % are scaled down if they would overflow the margins.
    \makeatletter
    \def\maxwidth{\ifdim\Gin@nat@width>\linewidth\linewidth
    \else\Gin@nat@width\fi}
    \makeatother
    \let\Oldincludegraphics\includegraphics
    % Set max figure width to be 80% of text width, for now hardcoded.
    \renewcommand{\includegraphics}[1]{\Oldincludegraphics[width=.8\maxwidth]{#1}}
    % Ensure that by default, figures have no caption (until we provide a
    % proper Figure object with a Caption API and a way to capture that
    % in the conversion process - todo).
    \usepackage{caption}
    \DeclareCaptionLabelFormat{nolabel}{}
    \captionsetup{labelformat=nolabel}

    \usepackage{adjustbox} % Used to constrain images to a maximum size 
    \usepackage{xcolor} % Allow colors to be defined
    \usepackage{enumerate} % Needed for markdown enumerations to work
    \usepackage{geometry} % Used to adjust the document margins
    \usepackage{amsmath} % Equations
    \usepackage{amssymb} % Equations
    \usepackage{textcomp} % defines textquotesingle
    % Hack from http://tex.stackexchange.com/a/47451/13684:
    \AtBeginDocument{%
        \def\PYZsq{\textquotesingle}% Upright quotes in Pygmentized code
    }
    \usepackage{upquote} % Upright quotes for verbatim code
    \usepackage{eurosym} % defines \euro
    \usepackage[mathletters]{ucs} % Extended unicode (utf-8) support
    \usepackage[utf8x]{inputenc} % Allow utf-8 characters in the tex document
    \usepackage{fancyvrb} % verbatim replacement that allows latex
    \usepackage{grffile} % extends the file name processing of package graphics 
                         % to support a larger range 
    % The hyperref package gives us a pdf with properly built
    % internal navigation ('pdf bookmarks' for the table of contents,
    % internal cross-reference links, web links for URLs, etc.)
    \usepackage{hyperref}
    \usepackage{longtable} % longtable support required by pandoc >1.10
    \usepackage{booktabs}  % table support for pandoc > 1.12.2
    \usepackage[inline]{enumitem} % IRkernel/repr support (it uses the enumerate* environment)
    \usepackage[normalem]{ulem} % ulem is needed to support strikethroughs (\sout)
                                % normalem makes italics be italics, not underlines
    


    
    %% lab_title.tplx %% 
 
\newcommand{\labtitle}{Lab02 Sifting Weather Data} 
    %% header_and_footer.tplx %%

% Header and Footer
\lhead{\textbf{\labtitle}}
\rhead{ENGR114 Engineering Programming}
\lfoot{Portland Community College, \the\year}
\cfoot{}
\rfoot{\thepage~of~\pageref{LastPage}}  % must compile twice for LastPage

%lines below header and above footer
\renewcommand{\headrulewidth}{0.4pt}
\renewcommand{\footrulewidth}{0.4pt}

% Tabs
\newcommand{\itab}[1]{\hspace{0em}\rlap{#1}}
\newcommand{\tab}[1]{\hspace{.4\textwidth}\rlap{#1}}
\newcommand{\tabA}[1]{\hspace{.2\textwidth}\rlap{#1}}
    %% title_sec_formatting.tplx %%

\titleformat{\section}[block]{\LARGE\bfseries\filcenter}{}{1em}{}

\titleformat{\subsection}[hang]{\Large\bfseries}{}{1em}{}
\titlespacing{\subsection}{-1.4em}{1.5em}{1em}

\titleformat{\subsubsection}[hang]{\large\bfseries}{}{1em}{}
\titlespacing{\subsubsection}{-1.1em}{1.5em}{0.8em}
    
        \title{Problem Solving 101 with Python}
        \author{Peter D. Kazarinoff, PhD}
        \date{}
    
    
    
    % Colors for the hyperref package
    \definecolor{urlcolor}{rgb}{0,.145,.698}
    \definecolor{linkcolor}{rgb}{.71,0.21,0.01}
    \definecolor{citecolor}{rgb}{.12,.54,.11}

    % ANSI colors
    \definecolor{ansi-black}{HTML}{3E424D}
    \definecolor{ansi-black-intense}{HTML}{282C36}
    \definecolor{ansi-red}{HTML}{E75C58}
    \definecolor{ansi-red-intense}{HTML}{B22B31}
    \definecolor{ansi-green}{HTML}{00A250}
    \definecolor{ansi-green-intense}{HTML}{007427}
    \definecolor{ansi-yellow}{HTML}{DDB62B}
    \definecolor{ansi-yellow-intense}{HTML}{B27D12}
    \definecolor{ansi-blue}{HTML}{208FFB}
    \definecolor{ansi-blue-intense}{HTML}{0065CA}
    \definecolor{ansi-magenta}{HTML}{D160C4}
    \definecolor{ansi-magenta-intense}{HTML}{A03196}
    \definecolor{ansi-cyan}{HTML}{60C6C8}
    \definecolor{ansi-cyan-intense}{HTML}{258F8F}
    \definecolor{ansi-white}{HTML}{C5C1B4}
    \definecolor{ansi-white-intense}{HTML}{A1A6B2}

    % commands and environments needed by pandoc snippets
    % extracted from the output of `pandoc -s`
    \providecommand{\tightlist}{%
      \setlength{\itemsep}{0pt}\setlength{\parskip}{0pt}}
    \DefineVerbatimEnvironment{Highlighting}{Verbatim}{commandchars=\\\{\}}
    % Add ',fontsize=\small' for more characters per line
    \newenvironment{Shaded}{}{}
    \newcommand{\KeywordTok}[1]{\textcolor[rgb]{0.00,0.44,0.13}{\textbf{{#1}}}}
    \newcommand{\DataTypeTok}[1]{\textcolor[rgb]{0.56,0.13,0.00}{{#1}}}
    \newcommand{\DecValTok}[1]{\textcolor[rgb]{0.25,0.63,0.44}{{#1}}}
    \newcommand{\BaseNTok}[1]{\textcolor[rgb]{0.25,0.63,0.44}{{#1}}}
    \newcommand{\FloatTok}[1]{\textcolor[rgb]{0.25,0.63,0.44}{{#1}}}
    \newcommand{\CharTok}[1]{\textcolor[rgb]{0.25,0.44,0.63}{{#1}}}
    \newcommand{\StringTok}[1]{\textcolor[rgb]{0.25,0.44,0.63}{{#1}}}
    \newcommand{\CommentTok}[1]{\textcolor[rgb]{0.38,0.63,0.69}{\textit{{#1}}}}
    \newcommand{\OtherTok}[1]{\textcolor[rgb]{0.00,0.44,0.13}{{#1}}}
    \newcommand{\AlertTok}[1]{\textcolor[rgb]{1.00,0.00,0.00}{\textbf{{#1}}}}
    \newcommand{\FunctionTok}[1]{\textcolor[rgb]{0.02,0.16,0.49}{{#1}}}
    \newcommand{\RegionMarkerTok}[1]{{#1}}
    \newcommand{\ErrorTok}[1]{\textcolor[rgb]{1.00,0.00,0.00}{\textbf{{#1}}}}
    \newcommand{\NormalTok}[1]{{#1}}
    
    % Additional commands for more recent versions of Pandoc
    \newcommand{\ConstantTok}[1]{\textcolor[rgb]{0.53,0.00,0.00}{{#1}}}
    \newcommand{\SpecialCharTok}[1]{\textcolor[rgb]{0.25,0.44,0.63}{{#1}}}
    \newcommand{\VerbatimStringTok}[1]{\textcolor[rgb]{0.25,0.44,0.63}{{#1}}}
    \newcommand{\SpecialStringTok}[1]{\textcolor[rgb]{0.73,0.40,0.53}{{#1}}}
    \newcommand{\ImportTok}[1]{{#1}}
    \newcommand{\DocumentationTok}[1]{\textcolor[rgb]{0.73,0.13,0.13}{\textit{{#1}}}}
    \newcommand{\AnnotationTok}[1]{\textcolor[rgb]{0.38,0.63,0.69}{\textbf{\textit{{#1}}}}}
    \newcommand{\CommentVarTok}[1]{\textcolor[rgb]{0.38,0.63,0.69}{\textbf{\textit{{#1}}}}}
    \newcommand{\VariableTok}[1]{\textcolor[rgb]{0.10,0.09,0.49}{{#1}}}
    \newcommand{\ControlFlowTok}[1]{\textcolor[rgb]{0.00,0.44,0.13}{\textbf{{#1}}}}
    \newcommand{\OperatorTok}[1]{\textcolor[rgb]{0.40,0.40,0.40}{{#1}}}
    \newcommand{\BuiltInTok}[1]{{#1}}
    \newcommand{\ExtensionTok}[1]{{#1}}
    \newcommand{\PreprocessorTok}[1]{\textcolor[rgb]{0.74,0.48,0.00}{{#1}}}
    \newcommand{\AttributeTok}[1]{\textcolor[rgb]{0.49,0.56,0.16}{{#1}}}
    \newcommand{\InformationTok}[1]{\textcolor[rgb]{0.38,0.63,0.69}{\textbf{\textit{{#1}}}}}
    \newcommand{\WarningTok}[1]{\textcolor[rgb]{0.38,0.63,0.69}{\textbf{\textit{{#1}}}}}
    
    
    % Define a nice break command that doesn't care if a line doesn't already
    % exist.
    \def\br{\hspace*{\fill} \\* }
    % Math Jax compatability definitions
    \def\gt{>}
    \def\lt{<}
    % Document parameters
    
        \title{Problem Solving 101 with Python}
        \author{Peter D. Kazarinoff, PhD}
        \date{}
    
    
    
    

    % Pygments definitions
    
\makeatletter
\def\PY@reset{\let\PY@it=\relax \let\PY@bf=\relax%
    \let\PY@ul=\relax \let\PY@tc=\relax%
    \let\PY@bc=\relax \let\PY@ff=\relax}
\def\PY@tok#1{\csname PY@tok@#1\endcsname}
\def\PY@toks#1+{\ifx\relax#1\empty\else%
    \PY@tok{#1}\expandafter\PY@toks\fi}
\def\PY@do#1{\PY@bc{\PY@tc{\PY@ul{%
    \PY@it{\PY@bf{\PY@ff{#1}}}}}}}
\def\PY#1#2{\PY@reset\PY@toks#1+\relax+\PY@do{#2}}

\expandafter\def\csname PY@tok@w\endcsname{\def\PY@tc##1{\textcolor[rgb]{0.73,0.73,0.73}{##1}}}
\expandafter\def\csname PY@tok@c\endcsname{\let\PY@it=\textit\def\PY@tc##1{\textcolor[rgb]{0.25,0.50,0.50}{##1}}}
\expandafter\def\csname PY@tok@cp\endcsname{\def\PY@tc##1{\textcolor[rgb]{0.74,0.48,0.00}{##1}}}
\expandafter\def\csname PY@tok@k\endcsname{\let\PY@bf=\textbf\def\PY@tc##1{\textcolor[rgb]{0.00,0.50,0.00}{##1}}}
\expandafter\def\csname PY@tok@kp\endcsname{\def\PY@tc##1{\textcolor[rgb]{0.00,0.50,0.00}{##1}}}
\expandafter\def\csname PY@tok@kt\endcsname{\def\PY@tc##1{\textcolor[rgb]{0.69,0.00,0.25}{##1}}}
\expandafter\def\csname PY@tok@o\endcsname{\def\PY@tc##1{\textcolor[rgb]{0.40,0.40,0.40}{##1}}}
\expandafter\def\csname PY@tok@ow\endcsname{\let\PY@bf=\textbf\def\PY@tc##1{\textcolor[rgb]{0.67,0.13,1.00}{##1}}}
\expandafter\def\csname PY@tok@nb\endcsname{\def\PY@tc##1{\textcolor[rgb]{0.00,0.50,0.00}{##1}}}
\expandafter\def\csname PY@tok@nf\endcsname{\def\PY@tc##1{\textcolor[rgb]{0.00,0.00,1.00}{##1}}}
\expandafter\def\csname PY@tok@nc\endcsname{\let\PY@bf=\textbf\def\PY@tc##1{\textcolor[rgb]{0.00,0.00,1.00}{##1}}}
\expandafter\def\csname PY@tok@nn\endcsname{\let\PY@bf=\textbf\def\PY@tc##1{\textcolor[rgb]{0.00,0.00,1.00}{##1}}}
\expandafter\def\csname PY@tok@ne\endcsname{\let\PY@bf=\textbf\def\PY@tc##1{\textcolor[rgb]{0.82,0.25,0.23}{##1}}}
\expandafter\def\csname PY@tok@nv\endcsname{\def\PY@tc##1{\textcolor[rgb]{0.10,0.09,0.49}{##1}}}
\expandafter\def\csname PY@tok@no\endcsname{\def\PY@tc##1{\textcolor[rgb]{0.53,0.00,0.00}{##1}}}
\expandafter\def\csname PY@tok@nl\endcsname{\def\PY@tc##1{\textcolor[rgb]{0.63,0.63,0.00}{##1}}}
\expandafter\def\csname PY@tok@ni\endcsname{\let\PY@bf=\textbf\def\PY@tc##1{\textcolor[rgb]{0.60,0.60,0.60}{##1}}}
\expandafter\def\csname PY@tok@na\endcsname{\def\PY@tc##1{\textcolor[rgb]{0.49,0.56,0.16}{##1}}}
\expandafter\def\csname PY@tok@nt\endcsname{\let\PY@bf=\textbf\def\PY@tc##1{\textcolor[rgb]{0.00,0.50,0.00}{##1}}}
\expandafter\def\csname PY@tok@nd\endcsname{\def\PY@tc##1{\textcolor[rgb]{0.67,0.13,1.00}{##1}}}
\expandafter\def\csname PY@tok@s\endcsname{\def\PY@tc##1{\textcolor[rgb]{0.73,0.13,0.13}{##1}}}
\expandafter\def\csname PY@tok@sd\endcsname{\let\PY@it=\textit\def\PY@tc##1{\textcolor[rgb]{0.73,0.13,0.13}{##1}}}
\expandafter\def\csname PY@tok@si\endcsname{\let\PY@bf=\textbf\def\PY@tc##1{\textcolor[rgb]{0.73,0.40,0.53}{##1}}}
\expandafter\def\csname PY@tok@se\endcsname{\let\PY@bf=\textbf\def\PY@tc##1{\textcolor[rgb]{0.73,0.40,0.13}{##1}}}
\expandafter\def\csname PY@tok@sr\endcsname{\def\PY@tc##1{\textcolor[rgb]{0.73,0.40,0.53}{##1}}}
\expandafter\def\csname PY@tok@ss\endcsname{\def\PY@tc##1{\textcolor[rgb]{0.10,0.09,0.49}{##1}}}
\expandafter\def\csname PY@tok@sx\endcsname{\def\PY@tc##1{\textcolor[rgb]{0.00,0.50,0.00}{##1}}}
\expandafter\def\csname PY@tok@m\endcsname{\def\PY@tc##1{\textcolor[rgb]{0.40,0.40,0.40}{##1}}}
\expandafter\def\csname PY@tok@gh\endcsname{\let\PY@bf=\textbf\def\PY@tc##1{\textcolor[rgb]{0.00,0.00,0.50}{##1}}}
\expandafter\def\csname PY@tok@gu\endcsname{\let\PY@bf=\textbf\def\PY@tc##1{\textcolor[rgb]{0.50,0.00,0.50}{##1}}}
\expandafter\def\csname PY@tok@gd\endcsname{\def\PY@tc##1{\textcolor[rgb]{0.63,0.00,0.00}{##1}}}
\expandafter\def\csname PY@tok@gi\endcsname{\def\PY@tc##1{\textcolor[rgb]{0.00,0.63,0.00}{##1}}}
\expandafter\def\csname PY@tok@gr\endcsname{\def\PY@tc##1{\textcolor[rgb]{1.00,0.00,0.00}{##1}}}
\expandafter\def\csname PY@tok@ge\endcsname{\let\PY@it=\textit}
\expandafter\def\csname PY@tok@gs\endcsname{\let\PY@bf=\textbf}
\expandafter\def\csname PY@tok@gp\endcsname{\let\PY@bf=\textbf\def\PY@tc##1{\textcolor[rgb]{0.00,0.00,0.50}{##1}}}
\expandafter\def\csname PY@tok@go\endcsname{\def\PY@tc##1{\textcolor[rgb]{0.53,0.53,0.53}{##1}}}
\expandafter\def\csname PY@tok@gt\endcsname{\def\PY@tc##1{\textcolor[rgb]{0.00,0.27,0.87}{##1}}}
\expandafter\def\csname PY@tok@err\endcsname{\def\PY@bc##1{\setlength{\fboxsep}{0pt}\fcolorbox[rgb]{1.00,0.00,0.00}{1,1,1}{\strut ##1}}}
\expandafter\def\csname PY@tok@kc\endcsname{\let\PY@bf=\textbf\def\PY@tc##1{\textcolor[rgb]{0.00,0.50,0.00}{##1}}}
\expandafter\def\csname PY@tok@kd\endcsname{\let\PY@bf=\textbf\def\PY@tc##1{\textcolor[rgb]{0.00,0.50,0.00}{##1}}}
\expandafter\def\csname PY@tok@kn\endcsname{\let\PY@bf=\textbf\def\PY@tc##1{\textcolor[rgb]{0.00,0.50,0.00}{##1}}}
\expandafter\def\csname PY@tok@kr\endcsname{\let\PY@bf=\textbf\def\PY@tc##1{\textcolor[rgb]{0.00,0.50,0.00}{##1}}}
\expandafter\def\csname PY@tok@bp\endcsname{\def\PY@tc##1{\textcolor[rgb]{0.00,0.50,0.00}{##1}}}
\expandafter\def\csname PY@tok@fm\endcsname{\def\PY@tc##1{\textcolor[rgb]{0.00,0.00,1.00}{##1}}}
\expandafter\def\csname PY@tok@vc\endcsname{\def\PY@tc##1{\textcolor[rgb]{0.10,0.09,0.49}{##1}}}
\expandafter\def\csname PY@tok@vg\endcsname{\def\PY@tc##1{\textcolor[rgb]{0.10,0.09,0.49}{##1}}}
\expandafter\def\csname PY@tok@vi\endcsname{\def\PY@tc##1{\textcolor[rgb]{0.10,0.09,0.49}{##1}}}
\expandafter\def\csname PY@tok@vm\endcsname{\def\PY@tc##1{\textcolor[rgb]{0.10,0.09,0.49}{##1}}}
\expandafter\def\csname PY@tok@sa\endcsname{\def\PY@tc##1{\textcolor[rgb]{0.73,0.13,0.13}{##1}}}
\expandafter\def\csname PY@tok@sb\endcsname{\def\PY@tc##1{\textcolor[rgb]{0.73,0.13,0.13}{##1}}}
\expandafter\def\csname PY@tok@sc\endcsname{\def\PY@tc##1{\textcolor[rgb]{0.73,0.13,0.13}{##1}}}
\expandafter\def\csname PY@tok@dl\endcsname{\def\PY@tc##1{\textcolor[rgb]{0.73,0.13,0.13}{##1}}}
\expandafter\def\csname PY@tok@s2\endcsname{\def\PY@tc##1{\textcolor[rgb]{0.73,0.13,0.13}{##1}}}
\expandafter\def\csname PY@tok@sh\endcsname{\def\PY@tc##1{\textcolor[rgb]{0.73,0.13,0.13}{##1}}}
\expandafter\def\csname PY@tok@s1\endcsname{\def\PY@tc##1{\textcolor[rgb]{0.73,0.13,0.13}{##1}}}
\expandafter\def\csname PY@tok@mb\endcsname{\def\PY@tc##1{\textcolor[rgb]{0.40,0.40,0.40}{##1}}}
\expandafter\def\csname PY@tok@mf\endcsname{\def\PY@tc##1{\textcolor[rgb]{0.40,0.40,0.40}{##1}}}
\expandafter\def\csname PY@tok@mh\endcsname{\def\PY@tc##1{\textcolor[rgb]{0.40,0.40,0.40}{##1}}}
\expandafter\def\csname PY@tok@mi\endcsname{\def\PY@tc##1{\textcolor[rgb]{0.40,0.40,0.40}{##1}}}
\expandafter\def\csname PY@tok@il\endcsname{\def\PY@tc##1{\textcolor[rgb]{0.40,0.40,0.40}{##1}}}
\expandafter\def\csname PY@tok@mo\endcsname{\def\PY@tc##1{\textcolor[rgb]{0.40,0.40,0.40}{##1}}}
\expandafter\def\csname PY@tok@ch\endcsname{\let\PY@it=\textit\def\PY@tc##1{\textcolor[rgb]{0.25,0.50,0.50}{##1}}}
\expandafter\def\csname PY@tok@cm\endcsname{\let\PY@it=\textit\def\PY@tc##1{\textcolor[rgb]{0.25,0.50,0.50}{##1}}}
\expandafter\def\csname PY@tok@cpf\endcsname{\let\PY@it=\textit\def\PY@tc##1{\textcolor[rgb]{0.25,0.50,0.50}{##1}}}
\expandafter\def\csname PY@tok@c1\endcsname{\let\PY@it=\textit\def\PY@tc##1{\textcolor[rgb]{0.25,0.50,0.50}{##1}}}
\expandafter\def\csname PY@tok@cs\endcsname{\let\PY@it=\textit\def\PY@tc##1{\textcolor[rgb]{0.25,0.50,0.50}{##1}}}

\def\PYZbs{\char`\\}
\def\PYZus{\char`\_}
\def\PYZob{\char`\{}
\def\PYZcb{\char`\}}
\def\PYZca{\char`\^}
\def\PYZam{\char`\&}
\def\PYZlt{\char`\<}
\def\PYZgt{\char`\>}
\def\PYZsh{\char`\#}
\def\PYZpc{\char`\%}
\def\PYZdl{\char`\$}
\def\PYZhy{\char`\-}
\def\PYZsq{\char`\'}
\def\PYZdq{\char`\"}
\def\PYZti{\char`\~}
% for compatibility with earlier versions
\def\PYZat{@}
\def\PYZlb{[}
\def\PYZrb{]}
\makeatother


    % Exact colors from NB
    \definecolor{incolor}{rgb}{0.0, 0.0, 0.5}
    \definecolor{outcolor}{rgb}{0.545, 0.0, 0.0}




    
    % Prevent overflowing lines due to hard-to-break entities
    \sloppy 
    % Setup hyperref package
    \hypersetup{
      breaklinks=true,  % so long urls are correctly broken across lines
      colorlinks=true,
      urlcolor=urlcolor,
      linkcolor=linkcolor,
      citecolor=citecolor,
      }
    % Slightly bigger margins than the latex defaults
    
    %% margins.tplx %%

% margins
\textwidth=7in
\textheight=9.0in
\topmargin=-0.5in
\headheight=15pt
\headsep=.5in
\hoffset = -0.5in

\pagestyle{fancy}

    

    \begin{document}
    
    
    

    
    

    
    \hypertarget{lab-02---sifting-weather-data}{%
\section{Lab 02 - Sifting Weather
Data}\label{lab-02---sifting-weather-data}}

    \hypertarget{prelab}{%
\subsection{Prelab}\label{prelab}}

You can find the relevant .xlsx-file posted along with these lab
instructions on D2L. After completing the in-class example, save your
script as a Jupyter notebook. Your work completing the in-class example
does not need to be submitted for credit.

    \hypertarget{lab}{%
\subsection{Lab}\label{lab}}

In this lab, you will be working with rainfall data that is stored in an
Excel file called \texttt{weather\_data.xlsx}. The Excel file contains
the rainfall on different days of the year in hundredths of an inch. The
rows of the Excel file represent months and the columns of the Excel
file represent days. The data stored in the cells of the Excel file
represent rainfall in hundredths of an inch.

You will use Python to derive some meaningful statistics from the
rainfall data.

    \hypertarget{import-external-packages}{%
\subsubsection{Import External
Packages}\label{import-external-packages}}

The first step in the lab is to add a header to your Jupyter Notebook
that includes the lab number and name, your name, the course, quarter
and date.

Below the header cell, you need to import \textbf{NumPy} and
\textbf{Pandas}. NumPy and Pandas are two external packages that you
will use in the lab. The standard alias \texttt{np} and \texttt{pd}
should be used for NumPy and Pandas respectively.

The Jupyter notebook code cell below shows the import lines.

    \begin{Verbatim}[commandchars=\\\{\}]
{\color{incolor}In [{\color{incolor}1}]:} \PY{k+kn}{import} \PY{n+nn}{numpy} \PY{k}{as} \PY{n+nn}{np}
        \PY{k+kn}{import} \PY{n+nn}{pandas} \PY{k}{as} \PY{n+nn}{pd}
\end{Verbatim}


    \hypertarget{save-the-weather-data-to-a-numpy-array}{%
\subsubsection{Save the weather data to a NumPy
array}\label{save-the-weather-data-to-a-numpy-array}}

The next step in the lab is to import the Excel file using Pandas and
save the weather data to a NumPy array.

Import the weather data from the \texttt{weather\_data.xls} file into a
Panda's dataframe called \texttt{df}. You can search the Pandas package
documentation help section to see how to use the
\texttt{pd.read\_excel()} function and why we need the keyword argument
\texttt{header=None} in this example.

Note: the .xlsx-file must be in your current file path, and quotes need
to surround the file name.

The Jupyter notebook code cell below shows the steps to import the
\texttt{weather\_data.xlsx} file into a Pandas dataframe, and save the
data in the variable \texttt{df}.

    \begin{Verbatim}[commandchars=\\\{\}]
{\color{incolor}In [{\color{incolor}2}]:} \PY{n}{df} \PY{o}{=} \PY{n}{pd}\PY{o}{.}\PY{n}{read\PYZus{}excel}\PY{p}{(}\PY{l+s+s1}{\PYZsq{}}\PY{l+s+s1}{weather\PYZus{}data.xlsx}\PY{l+s+s1}{\PYZsq{}}\PY{p}{,} \PY{n}{header}\PY{o}{=}\PY{k+kc}{None}\PY{p}{)}
\end{Verbatim}


    You can use Pandas methods to view the top of the dataframe with
\texttt{df.head()} and \texttt{df.tail()}. When first imported, the
dataframe \texttt{df} contains a couple values equal to \texttt{-99999}.
Running the command \texttt{print(df.shape)} will indicate the dataframe
\texttt{df} has dimensions of 12 rows x 31 columns. Also check the value
in row 1 (index 0), column 3 (index 2) in the dataframe is \texttt{272}
by running the \texttt{wdf.head()} method.

The code cell below completes these steps.

    \begin{Verbatim}[commandchars=\\\{\}]
{\color{incolor}In [{\color{incolor}3}]:} \PY{n+nb}{print}\PY{p}{(}\PY{n}{df}\PY{o}{.}\PY{n}{shape}\PY{p}{)}
        \PY{n}{df}
\end{Verbatim}


    \begin{Verbatim}[commandchars=\\\{\}]
(12, 31)

    \end{Verbatim}

\begin{Verbatim}[commandchars=\\\{\}]
{\color{outcolor}Out[{\color{outcolor}3}]:}      0    1    2   3    4    5   6   7   8    9   {\ldots}    21  22   23   24  \textbackslash{}
        0     0    0  272   0    0    0   0   9  26    1  {\ldots}     1  15  224    0   
        1    61  103    0   2    0    0   0   2   0   68  {\ldots}     0   0    5    0   
        2     2    0   17  27    0   10   0   0  30   25  {\ldots}    26   0    0    9   
        3   260    1    0   0    0    1   3   0   5    0  {\ldots}     0   0    0    0   
        4    47    0    0   0    5  115  49  81   0    0  {\ldots}     0   0    0    0   
        5     0    0   30  42    0    0   0   0   0    0  {\ldots}     0   2    2  240   
        6     0    0    0   0    0    0   0   7   0    0  {\ldots}     0   0    0   10   
        7     0   45    0   0    0    0   0   0   5    0  {\ldots}     0   0  158   62   
        8     0    0    0   0    0    0   0   0   0    0  {\ldots}    37   0    0    0   
        9     0    0    0  14  156    0   0   0   0  110  {\ldots}     0   0    0    0   
        10    1  163    5   0    0    0   0   0   0    0  {\ldots}     0   2    2   10   
        11    0    0    0   0    0   45   0   0   0   15  {\ldots}    34   0    0    0   
        
             25  26   27     28     29     30  
        0     0   0    0      0     33     33  
        1     0   0   62 -99999 -99999 -99999  
        2     6  78    0      5      8      0  
        3     9  35   13     86      0 -99999  
        4     0   2    0      0      0      0  
        5    16  35   14     14      8 -99999  
        6     0   0    5      0      0      0  
        7     1   0    0      0      0      0  
        8     0   6  138     58     10 -99999  
        9     0   0    0      0      0      1  
        10  255   2    0      0      0 -99999  
        11    0   0    0      0      0      0  
        
        [12 rows x 31 columns]
\end{Verbatim}
            
    Next, convert the Pandas dataframe \texttt{df} to a NumPy array called
\texttt{wd} (for weather data). Use NumPy's \texttt{np.array()}
function.

The Jupyter notebook code cell below demonstrates this functionality.

    \begin{Verbatim}[commandchars=\\\{\}]
{\color{incolor}In [{\color{incolor}4}]:} \PY{n}{wd} \PY{o}{=} \PY{n}{np}\PY{o}{.}\PY{n}{array}\PY{p}{(}\PY{n}{df}\PY{p}{)}
\end{Verbatim}
\newpage

    Next use Python's \texttt{type()} function to verify the variable
\texttt{wd} is a NumPy array. Also verify that the NumPy array is 5 by
31, which is the same dimensions (how many rows, how many columns) as
the dataframe \texttt{df} that we previously printed out.

The code cell below prints \texttt{wd}'s data type and prints the
dimensions of \texttt{wd}

    \begin{Verbatim}[commandchars=\\\{\}]
{\color{incolor}In [{\color{incolor}5}]:} \PY{n+nb}{print}\PY{p}{(}\PY{n+nb}{type}\PY{p}{(}\PY{n}{wd}\PY{p}{)}\PY{p}{)}
        \PY{n+nb}{print}\PY{p}{(}\PY{n}{wd}\PY{o}{.}\PY{n}{shape}\PY{p}{)}
\end{Verbatim}


    \begin{Verbatim}[commandchars=\\\{\}]
<class 'numpy.ndarray'>
(12, 31)
\end{Verbatim}

    \hypertarget{clean-up-the-data}{%
\subsubsection{Clean up the data}\label{clean-up-the-data}}

The weather data is in hundredths of an inch. Multiply each value in the
array by \texttt{0.01} so that the array contains rain fall in inches
(not hundredths of an inch).

The code cell below shows how to convert the weather data from
hundredths of an inch to inches. When the multiplication symbol
\texttt{*} is used with a NumPy array, the multiplication is
element-wise. This means each value in the array is multiplied by the
scalar ``0.01```.

\begin{Verbatim}[commandchars=\\\{\}]
{\color{incolor}In [{\color{incolor}6}]:} \PY{n}{wd} \PY{o}{=} \PY{n}{wd}\PY{o}{*}\PY{l+m+mf}{0.01}
\end{Verbatim}


    \hypertarget{define-a-month-array-and-remove--99999-values}{%
\subsubsection{Define a month array and remove -99999
values}\label{define-a-month-array-and-remove--99999-values}}

Define a variable called \texttt{m} and set it equal to \texttt{1} for
now (Month \texttt{1} is February, remember Python counting starts at
0). \texttt{m} represents the month for which we will compute
statistics. Later, you will change \texttt{m} to other values.
\texttt{0} means January, \texttt{1} means February, and so on.

Use the value in \texttt{m} to extract that month's data from your
\texttt{wd} array. Store these values in an array called \texttt{md}
(for month data). Since February has 28 days, you will see positive
numbers, zeros and \texttt{NaN} present in the array.

For example, according to our data, the rainfall for the first 3 days in
February are \texttt{0.61} inches, \texttt{1.03} inches, and \texttt{0}
inches.

The Jupyter notebook cell which shows this operation is below. Note how
indexing is used to pull out all of the values from one row. The general
syntax for indexing a NumPy array is \texttt{array{[}row,col{]}}. If we
use \texttt{wd{[}1,:{]}}, this means \texttt{row=1}, \texttt{col=all}
(row one, all columns). Note the values \texttt{-9.9999} present in the
array \texttt{md}.

    \begin{Verbatim}[commandchars=\\\{\}]
{\color{incolor}In [{\color{incolor}7}]:} \PY{n}{m} \PY{o}{=} \PY{l+m+mi}{1}
        \PY{n}{md} \PY{o}{=} \PY{n}{wd}\PY{p}{[}\PY{l+m+mi}{1}\PY{p}{,}\PY{p}{:}\PY{p}{]}
        \PY{n+nb}{print}\PY{p}{(}\PY{n}{md}\PY{p}{)}
\end{Verbatim}


    \begin{Verbatim}[commandchars=\\\{\}]
[ 6.1000e-01  1.0300e+00  0.0000e+00  2.0000e-02  0.0000e+00  0.0000e+00
  0.0000e+00  2.0000e-02  0.0000e+00  6.8000e-01  3.0000e-02  3.0000e-02
  1.0000e-02  0.0000e+00  0.0000e+00  0.0000e+00  1.0000e-02  1.3500e+00
  3.0000e-02  6.6000e-01  0.0000e+00  0.0000e+00  0.0000e+00  5.0000e-02
  0.0000e+00  0.0000e+00  0.0000e+00  6.2000e-01 -9.9999e+02 -9.9999e+02
 -9.9999e+02]

    \end{Verbatim}

    The next step is to remove all of the \texttt{-9.9999} values from the
\texttt{md} array. If we find the average of the values in \texttt{md}
now, which corresponds to the average rainfall in February, the value
that's returned doesn't make sense. There is no way the rain fall could
be negative.

The code cell below shows how to find the average of the values in the
array \texttt{md}. Note that \texttt{-9.9999} values are present in the
array.

    \begin{Verbatim}[commandchars=\\\{\}]
{\color{incolor}In [{\color{incolor}8}]:} \PY{n}{np}\PY{o}{.}\PY{n}{mean}\PY{p}{(}\PY{n}{md}\PY{p}{)}
\end{Verbatim}


\begin{Verbatim}[commandchars=\\\{\}]
{\color{outcolor}Out[{\color{outcolor}8}]:} -96.60709677419354
\end{Verbatim}
            
    To remove the -9.9999 values from the array, use a boolean mask.

The command \texttt{md\textgreater{}-1} creates a boolean mask. The
boolean mask is an array the same size as \texttt{md}, but only contains
\texttt{True} or \texttt{False} as values. In the mask, all locations
where \texttt{md\textgreater{}-1} are set to \texttt{True}; all other
values in the boolean mask are set to \texttt{False} (We use
\texttt{md\textgreater{}-1} instead of \texttt{md\textgreater{}0}
because we want to keep the \texttt{0} values, we just want to get rid
of the \texttt{-9.9999} values).

The code cell below produces the boolean mask. Note how the last three
elements are \texttt{False}. The \texttt{False} values are in the same
location as the \texttt{-9.9999} values in \texttt{md}.

    \begin{Verbatim}[commandchars=\\\{\}]
{\color{incolor}In [{\color{incolor}9}]:} \PY{n}{md}\PY{o}{\PYZgt{}}\PY{o}{\PYZhy{}}\PY{l+m+mi}{1}
\end{Verbatim}


\begin{Verbatim}[commandchars=\\\{\}]
{\color{outcolor}Out[{\color{outcolor}9}]:} array([ True,  True,  True,  True,  True,  True,  True,  True,  True,
                True,  True,  True,  True,  True,  True,  True,  True,  True,
                True,  True,  True,  True,  True,  True,  True,  True,  True,
                True, False, False, False])
\end{Verbatim}
            
    We then use the boolean mask \texttt{md\textgreater{}-1} to index the
array \texttt{md}. All the locations in \texttt{wd} where the boolean
mask is set to \texttt{True} are pull out. All the locations in
\texttt{wd} where the boolean mask is \texttt{False} do not get pulled
out. We put the results of the indexing operation back into the variable
\texttt{wd} to be used later.

The code cell below creates the boolean mask, indexes out values from
\texttt{wd} based on the mask, and puts the results back into
\texttt{wd}. As a result, \texttt{wd} does not contain \texttt{-9.9999} in any
location.

    \begin{Verbatim}[commandchars=\\\{\}]
{\color{incolor}In [{\color{incolor}10}]:} \PY{n}{mask} \PY{o}{=} \PY{n}{md}\PY{o}{\PYZgt{}}\PY{o}{\PYZhy{}}\PY{l+m+mi}{1}
         \PY{n}{md} \PY{o}{=} \PY{n}{md}\PY{p}{[}\PY{n}{mask}\PY{p}{]}
         \PY{n+nb}{print}\PY{p}{(}\PY{n}{md}\PY{p}{)}
\end{Verbatim}


    \begin{Verbatim}[commandchars=\\\{\}]
[0.61 1.03 0.   0.02 0.   0.   0.   0.02 0.   0.68 0.03 0.03 0.01 0.
 0.   0.   0.01 1.35 0.03 0.66 0.   0.   0.   0.05 0.   0.   0.   0.62]

    \end{Verbatim}
\newpage

    \hypertarget{compute-statistics-for-the-choosen-month}{%
\subsubsection{Compute Statistics for the choosen
month}\label{compute-statistics-for-the-choosen-month}}

The last part of the lab involves computing and storing the following
statistics/properties of the data in the \texttt{md} array in variables
named as follows:

\begin{itemize}
\tightlist
\item
  \texttt{m\_avg}: average inches fallen in month \texttt{m}
\item
  \texttt{m\_max}: maximum inches fallen in month \texttt{m}
\item
  \texttt{d\_max}: day(s) of month \texttt{m} in which \texttt{m\_max}
  inches fell
\item
  \texttt{nd\_zero}: number of day(s) of month \texttt{m} in which zero
  inches fell
\item
  \texttt{d\_zero}: day(s) of month \texttt{m} in which zero inches fell
\item
  \texttt{d\_nonzero}: day(s) of month \texttt{m} in which at least some
  rain fell
\item
  \texttt{nd\_nonzero}: number of day(s) in month \texttt{m} in which at
  least some rain fell\\
\item
  \texttt{m\_avg\_nonzero}: average inches of only day(s) in which some
  rain fell
\end{itemize}

NumPy's \texttt{np.where()} function is useful here as are
\texttt{np.mean()}, \texttt{np.max()} and an array's \texttt{array.size}
attribute.

    Test all of your values for the month of February manually to make sure
they are correct.

Then run your script for the January data (\texttt{m=0}) and check the
results against manual calculations for that month.

You may wish to test your script against other month's data. When you
are convinced your results and variable names are correct, set
\texttt{m=1} and restart the Jupyter notebook Kernel and clear all
output. Run each cell one last time and ensure the script is error free.

For your own records, save your \texttt{lab2.ipynb} file and your
\texttt{weather\_data.xlsx} file to the same directory. You do not need
to submit your \texttt{weather\_data.xlsx} file because your instructor
has their own copy to test your script.

    \hypertarget{deliverable}{%
\subsection{Deliverable}\label{deliverable}}

One .ipynb-file with all cells run. The .ipynb-file should run without
errors. Uploaded the .ipynb file to D2L by the end of the lab period.

    \hypertarget{by-d.-kruger-adapted-by-p.-kazarinoff-portland-community-college-2019}{%
\paragraph{\texorpdfstring{\emph{By D. Kruger, adapted by P. Kazarinoff,
Portland Community College,
2019}}{By D. Kruger, adapted by P. Kazarinoff, Portland Community College, 2019}}\label{by-d.-kruger-adapted-by-p.-kazarinoff-portland-community-college-2019}}


    % Add a bibliography block to the postdoc
    
    
    
    \end{document}

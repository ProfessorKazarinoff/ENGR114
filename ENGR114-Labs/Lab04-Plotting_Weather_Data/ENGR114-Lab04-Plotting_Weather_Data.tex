%% ENGR114_lab_assignment.tplx %%
%
% Built off of the article.tplx template %


% Default to the notebook output style

    


% Inherit from the specified cell style.




    
    \documentclass[11pt]{article}

    
    
    %% installed packages_rev2.tplx %%

\usepackage{fancyhdr}
\usepackage{lastpage}
\usepackage{framed,color}
\definecolor{shadecolor}{rgb}{.8,.8,.8}
\usepackage{titlesec}
% no indent on any paragraphs, vertical spacing between paragraphs is set to 1em
\usepackage[]{parskip}  % add [skip=1em] if the compiler will allow.

% for MATLAB syntax highlighting
\usepackage{listings}             % Include the listings-package
\definecolor{mygray}{rgb}{0.8,0.8,0.8} % color values Red, Green, Blue
\definecolor{mygreen}{RGB}{28,172,0}
\definecolor{mylilas}{RGB}{170,55,241}
    
    \usepackage[T1]{fontenc}
    % Nicer default font (+ math font) than Computer Modern for most use cases
    \usepackage{mathpazo}

    % Basic figure setup, for now with no caption control since it's done
    % automatically by Pandoc (which extracts ![](path) syntax from Markdown).
    \usepackage{graphicx}
    % We will generate all images so they have a width \maxwidth. This means
    % that they will get their normal width if they fit onto the page, but
    % are scaled down if they would overflow the margins.
    \makeatletter
    \def\maxwidth{\ifdim\Gin@nat@width>\linewidth\linewidth
    \else\Gin@nat@width\fi}
    \makeatother
    \let\Oldincludegraphics\includegraphics
    % Set max figure width to be 80% of text width, for now hardcoded.
    \renewcommand{\includegraphics}[1]{\Oldincludegraphics[width=.8\maxwidth]{#1}}
    % Ensure that by default, figures have no caption (until we provide a
    % proper Figure object with a Caption API and a way to capture that
    % in the conversion process - todo).
    \usepackage{caption}
    \DeclareCaptionLabelFormat{nolabel}{}
    \captionsetup{labelformat=nolabel}

    \usepackage{adjustbox} % Used to constrain images to a maximum size 
    \usepackage{xcolor} % Allow colors to be defined
    \usepackage{enumerate} % Needed for markdown enumerations to work
    \usepackage{geometry} % Used to adjust the document margins
    \usepackage{amsmath} % Equations
    \usepackage{amssymb} % Equations
    \usepackage{textcomp} % defines textquotesingle
    % Hack from http://tex.stackexchange.com/a/47451/13684:
    \AtBeginDocument{%
        \def\PYZsq{\textquotesingle}% Upright quotes in Pygmentized code
    }
    \usepackage{upquote} % Upright quotes for verbatim code
    \usepackage{eurosym} % defines \euro
    \usepackage[mathletters]{ucs} % Extended unicode (utf-8) support
    \usepackage[utf8x]{inputenc} % Allow utf-8 characters in the tex document
    \usepackage{fancyvrb} % verbatim replacement that allows latex
    \usepackage{grffile} % extends the file name processing of package graphics 
                         % to support a larger range 
    % The hyperref package gives us a pdf with properly built
    % internal navigation ('pdf bookmarks' for the table of contents,
    % internal cross-reference links, web links for URLs, etc.)
    \usepackage{hyperref}
    \usepackage{longtable} % longtable support required by pandoc >1.10
    \usepackage{booktabs}  % table support for pandoc > 1.12.2
    \usepackage[inline]{enumitem} % IRkernel/repr support (it uses the enumerate* environment)
    \usepackage[normalem]{ulem} % ulem is needed to support strikethroughs (\sout)
                                % normalem makes italics be italics, not underlines
    


    
    %% lab_title.tplx %% 
 
\newcommand{\labtitle}{Lab04 Plotting Weather Data} 
    %% header_and_footer.tplx %%

% Header and Footer
\lhead{\textbf{\labtitle}}
\rhead{ENGR114 Engineering Programming}
\lfoot{Portland Community College, \the\year}
\cfoot{}
\rfoot{\thepage~of~\pageref{LastPage}}  % must compile twice for LastPage

%lines below header and above footer
\renewcommand{\headrulewidth}{0.4pt}
\renewcommand{\footrulewidth}{0.4pt}

% Tabs
\newcommand{\itab}[1]{\hspace{0em}\rlap{#1}}
\newcommand{\tab}[1]{\hspace{.4\textwidth}\rlap{#1}}
\newcommand{\tabA}[1]{\hspace{.2\textwidth}\rlap{#1}}
    %% title_sec_formatting.tplx %%

\titleformat{\section}[block]{\LARGE\bfseries\filcenter}{}{1em}{}

\titleformat{\subsection}[hang]{\Large\bfseries}{}{1em}{}
\titlespacing{\subsection}{-1.4em}{1.5em}{1em}

\titleformat{\subsubsection}[hang]{\large\bfseries}{}{1em}{}
\titlespacing{\subsubsection}{-1.1em}{1.5em}{0.8em}
    
        \title{Problem Solving 101 with Python}
        \author{Peter D. Kazarinoff, PhD}
        \date{}
    
    
    
    % Colors for the hyperref package
    \definecolor{urlcolor}{rgb}{0,.145,.698}
    \definecolor{linkcolor}{rgb}{.71,0.21,0.01}
    \definecolor{citecolor}{rgb}{.12,.54,.11}

    % ANSI colors
    \definecolor{ansi-black}{HTML}{3E424D}
    \definecolor{ansi-black-intense}{HTML}{282C36}
    \definecolor{ansi-red}{HTML}{E75C58}
    \definecolor{ansi-red-intense}{HTML}{B22B31}
    \definecolor{ansi-green}{HTML}{00A250}
    \definecolor{ansi-green-intense}{HTML}{007427}
    \definecolor{ansi-yellow}{HTML}{DDB62B}
    \definecolor{ansi-yellow-intense}{HTML}{B27D12}
    \definecolor{ansi-blue}{HTML}{208FFB}
    \definecolor{ansi-blue-intense}{HTML}{0065CA}
    \definecolor{ansi-magenta}{HTML}{D160C4}
    \definecolor{ansi-magenta-intense}{HTML}{A03196}
    \definecolor{ansi-cyan}{HTML}{60C6C8}
    \definecolor{ansi-cyan-intense}{HTML}{258F8F}
    \definecolor{ansi-white}{HTML}{C5C1B4}
    \definecolor{ansi-white-intense}{HTML}{A1A6B2}

    % commands and environments needed by pandoc snippets
    % extracted from the output of `pandoc -s`
    \providecommand{\tightlist}{%
      \setlength{\itemsep}{0pt}\setlength{\parskip}{0pt}}
    \DefineVerbatimEnvironment{Highlighting}{Verbatim}{commandchars=\\\{\}}
    % Add ',fontsize=\small' for more characters per line
    \newenvironment{Shaded}{}{}
    \newcommand{\KeywordTok}[1]{\textcolor[rgb]{0.00,0.44,0.13}{\textbf{{#1}}}}
    \newcommand{\DataTypeTok}[1]{\textcolor[rgb]{0.56,0.13,0.00}{{#1}}}
    \newcommand{\DecValTok}[1]{\textcolor[rgb]{0.25,0.63,0.44}{{#1}}}
    \newcommand{\BaseNTok}[1]{\textcolor[rgb]{0.25,0.63,0.44}{{#1}}}
    \newcommand{\FloatTok}[1]{\textcolor[rgb]{0.25,0.63,0.44}{{#1}}}
    \newcommand{\CharTok}[1]{\textcolor[rgb]{0.25,0.44,0.63}{{#1}}}
    \newcommand{\StringTok}[1]{\textcolor[rgb]{0.25,0.44,0.63}{{#1}}}
    \newcommand{\CommentTok}[1]{\textcolor[rgb]{0.38,0.63,0.69}{\textit{{#1}}}}
    \newcommand{\OtherTok}[1]{\textcolor[rgb]{0.00,0.44,0.13}{{#1}}}
    \newcommand{\AlertTok}[1]{\textcolor[rgb]{1.00,0.00,0.00}{\textbf{{#1}}}}
    \newcommand{\FunctionTok}[1]{\textcolor[rgb]{0.02,0.16,0.49}{{#1}}}
    \newcommand{\RegionMarkerTok}[1]{{#1}}
    \newcommand{\ErrorTok}[1]{\textcolor[rgb]{1.00,0.00,0.00}{\textbf{{#1}}}}
    \newcommand{\NormalTok}[1]{{#1}}
    
    % Additional commands for more recent versions of Pandoc
    \newcommand{\ConstantTok}[1]{\textcolor[rgb]{0.53,0.00,0.00}{{#1}}}
    \newcommand{\SpecialCharTok}[1]{\textcolor[rgb]{0.25,0.44,0.63}{{#1}}}
    \newcommand{\VerbatimStringTok}[1]{\textcolor[rgb]{0.25,0.44,0.63}{{#1}}}
    \newcommand{\SpecialStringTok}[1]{\textcolor[rgb]{0.73,0.40,0.53}{{#1}}}
    \newcommand{\ImportTok}[1]{{#1}}
    \newcommand{\DocumentationTok}[1]{\textcolor[rgb]{0.73,0.13,0.13}{\textit{{#1}}}}
    \newcommand{\AnnotationTok}[1]{\textcolor[rgb]{0.38,0.63,0.69}{\textbf{\textit{{#1}}}}}
    \newcommand{\CommentVarTok}[1]{\textcolor[rgb]{0.38,0.63,0.69}{\textbf{\textit{{#1}}}}}
    \newcommand{\VariableTok}[1]{\textcolor[rgb]{0.10,0.09,0.49}{{#1}}}
    \newcommand{\ControlFlowTok}[1]{\textcolor[rgb]{0.00,0.44,0.13}{\textbf{{#1}}}}
    \newcommand{\OperatorTok}[1]{\textcolor[rgb]{0.40,0.40,0.40}{{#1}}}
    \newcommand{\BuiltInTok}[1]{{#1}}
    \newcommand{\ExtensionTok}[1]{{#1}}
    \newcommand{\PreprocessorTok}[1]{\textcolor[rgb]{0.74,0.48,0.00}{{#1}}}
    \newcommand{\AttributeTok}[1]{\textcolor[rgb]{0.49,0.56,0.16}{{#1}}}
    \newcommand{\InformationTok}[1]{\textcolor[rgb]{0.38,0.63,0.69}{\textbf{\textit{{#1}}}}}
    \newcommand{\WarningTok}[1]{\textcolor[rgb]{0.38,0.63,0.69}{\textbf{\textit{{#1}}}}}
    
    
    % Define a nice break command that doesn't care if a line doesn't already
    % exist.
    \def\br{\hspace*{\fill} \\* }
    % Math Jax compatability definitions
    \def\gt{>}
    \def\lt{<}
    % Document parameters
    
        \title{Problem Solving 101 with Python}
        \author{Peter D. Kazarinoff, PhD}
        \date{}
    
    
    
    

    % Pygments definitions
    
\makeatletter
\def\PY@reset{\let\PY@it=\relax \let\PY@bf=\relax%
    \let\PY@ul=\relax \let\PY@tc=\relax%
    \let\PY@bc=\relax \let\PY@ff=\relax}
\def\PY@tok#1{\csname PY@tok@#1\endcsname}
\def\PY@toks#1+{\ifx\relax#1\empty\else%
    \PY@tok{#1}\expandafter\PY@toks\fi}
\def\PY@do#1{\PY@bc{\PY@tc{\PY@ul{%
    \PY@it{\PY@bf{\PY@ff{#1}}}}}}}
\def\PY#1#2{\PY@reset\PY@toks#1+\relax+\PY@do{#2}}

\expandafter\def\csname PY@tok@w\endcsname{\def\PY@tc##1{\textcolor[rgb]{0.73,0.73,0.73}{##1}}}
\expandafter\def\csname PY@tok@c\endcsname{\let\PY@it=\textit\def\PY@tc##1{\textcolor[rgb]{0.25,0.50,0.50}{##1}}}
\expandafter\def\csname PY@tok@cp\endcsname{\def\PY@tc##1{\textcolor[rgb]{0.74,0.48,0.00}{##1}}}
\expandafter\def\csname PY@tok@k\endcsname{\let\PY@bf=\textbf\def\PY@tc##1{\textcolor[rgb]{0.00,0.50,0.00}{##1}}}
\expandafter\def\csname PY@tok@kp\endcsname{\def\PY@tc##1{\textcolor[rgb]{0.00,0.50,0.00}{##1}}}
\expandafter\def\csname PY@tok@kt\endcsname{\def\PY@tc##1{\textcolor[rgb]{0.69,0.00,0.25}{##1}}}
\expandafter\def\csname PY@tok@o\endcsname{\def\PY@tc##1{\textcolor[rgb]{0.40,0.40,0.40}{##1}}}
\expandafter\def\csname PY@tok@ow\endcsname{\let\PY@bf=\textbf\def\PY@tc##1{\textcolor[rgb]{0.67,0.13,1.00}{##1}}}
\expandafter\def\csname PY@tok@nb\endcsname{\def\PY@tc##1{\textcolor[rgb]{0.00,0.50,0.00}{##1}}}
\expandafter\def\csname PY@tok@nf\endcsname{\def\PY@tc##1{\textcolor[rgb]{0.00,0.00,1.00}{##1}}}
\expandafter\def\csname PY@tok@nc\endcsname{\let\PY@bf=\textbf\def\PY@tc##1{\textcolor[rgb]{0.00,0.00,1.00}{##1}}}
\expandafter\def\csname PY@tok@nn\endcsname{\let\PY@bf=\textbf\def\PY@tc##1{\textcolor[rgb]{0.00,0.00,1.00}{##1}}}
\expandafter\def\csname PY@tok@ne\endcsname{\let\PY@bf=\textbf\def\PY@tc##1{\textcolor[rgb]{0.82,0.25,0.23}{##1}}}
\expandafter\def\csname PY@tok@nv\endcsname{\def\PY@tc##1{\textcolor[rgb]{0.10,0.09,0.49}{##1}}}
\expandafter\def\csname PY@tok@no\endcsname{\def\PY@tc##1{\textcolor[rgb]{0.53,0.00,0.00}{##1}}}
\expandafter\def\csname PY@tok@nl\endcsname{\def\PY@tc##1{\textcolor[rgb]{0.63,0.63,0.00}{##1}}}
\expandafter\def\csname PY@tok@ni\endcsname{\let\PY@bf=\textbf\def\PY@tc##1{\textcolor[rgb]{0.60,0.60,0.60}{##1}}}
\expandafter\def\csname PY@tok@na\endcsname{\def\PY@tc##1{\textcolor[rgb]{0.49,0.56,0.16}{##1}}}
\expandafter\def\csname PY@tok@nt\endcsname{\let\PY@bf=\textbf\def\PY@tc##1{\textcolor[rgb]{0.00,0.50,0.00}{##1}}}
\expandafter\def\csname PY@tok@nd\endcsname{\def\PY@tc##1{\textcolor[rgb]{0.67,0.13,1.00}{##1}}}
\expandafter\def\csname PY@tok@s\endcsname{\def\PY@tc##1{\textcolor[rgb]{0.73,0.13,0.13}{##1}}}
\expandafter\def\csname PY@tok@sd\endcsname{\let\PY@it=\textit\def\PY@tc##1{\textcolor[rgb]{0.73,0.13,0.13}{##1}}}
\expandafter\def\csname PY@tok@si\endcsname{\let\PY@bf=\textbf\def\PY@tc##1{\textcolor[rgb]{0.73,0.40,0.53}{##1}}}
\expandafter\def\csname PY@tok@se\endcsname{\let\PY@bf=\textbf\def\PY@tc##1{\textcolor[rgb]{0.73,0.40,0.13}{##1}}}
\expandafter\def\csname PY@tok@sr\endcsname{\def\PY@tc##1{\textcolor[rgb]{0.73,0.40,0.53}{##1}}}
\expandafter\def\csname PY@tok@ss\endcsname{\def\PY@tc##1{\textcolor[rgb]{0.10,0.09,0.49}{##1}}}
\expandafter\def\csname PY@tok@sx\endcsname{\def\PY@tc##1{\textcolor[rgb]{0.00,0.50,0.00}{##1}}}
\expandafter\def\csname PY@tok@m\endcsname{\def\PY@tc##1{\textcolor[rgb]{0.40,0.40,0.40}{##1}}}
\expandafter\def\csname PY@tok@gh\endcsname{\let\PY@bf=\textbf\def\PY@tc##1{\textcolor[rgb]{0.00,0.00,0.50}{##1}}}
\expandafter\def\csname PY@tok@gu\endcsname{\let\PY@bf=\textbf\def\PY@tc##1{\textcolor[rgb]{0.50,0.00,0.50}{##1}}}
\expandafter\def\csname PY@tok@gd\endcsname{\def\PY@tc##1{\textcolor[rgb]{0.63,0.00,0.00}{##1}}}
\expandafter\def\csname PY@tok@gi\endcsname{\def\PY@tc##1{\textcolor[rgb]{0.00,0.63,0.00}{##1}}}
\expandafter\def\csname PY@tok@gr\endcsname{\def\PY@tc##1{\textcolor[rgb]{1.00,0.00,0.00}{##1}}}
\expandafter\def\csname PY@tok@ge\endcsname{\let\PY@it=\textit}
\expandafter\def\csname PY@tok@gs\endcsname{\let\PY@bf=\textbf}
\expandafter\def\csname PY@tok@gp\endcsname{\let\PY@bf=\textbf\def\PY@tc##1{\textcolor[rgb]{0.00,0.00,0.50}{##1}}}
\expandafter\def\csname PY@tok@go\endcsname{\def\PY@tc##1{\textcolor[rgb]{0.53,0.53,0.53}{##1}}}
\expandafter\def\csname PY@tok@gt\endcsname{\def\PY@tc##1{\textcolor[rgb]{0.00,0.27,0.87}{##1}}}
\expandafter\def\csname PY@tok@err\endcsname{\def\PY@bc##1{\setlength{\fboxsep}{0pt}\fcolorbox[rgb]{1.00,0.00,0.00}{1,1,1}{\strut ##1}}}
\expandafter\def\csname PY@tok@kc\endcsname{\let\PY@bf=\textbf\def\PY@tc##1{\textcolor[rgb]{0.00,0.50,0.00}{##1}}}
\expandafter\def\csname PY@tok@kd\endcsname{\let\PY@bf=\textbf\def\PY@tc##1{\textcolor[rgb]{0.00,0.50,0.00}{##1}}}
\expandafter\def\csname PY@tok@kn\endcsname{\let\PY@bf=\textbf\def\PY@tc##1{\textcolor[rgb]{0.00,0.50,0.00}{##1}}}
\expandafter\def\csname PY@tok@kr\endcsname{\let\PY@bf=\textbf\def\PY@tc##1{\textcolor[rgb]{0.00,0.50,0.00}{##1}}}
\expandafter\def\csname PY@tok@bp\endcsname{\def\PY@tc##1{\textcolor[rgb]{0.00,0.50,0.00}{##1}}}
\expandafter\def\csname PY@tok@fm\endcsname{\def\PY@tc##1{\textcolor[rgb]{0.00,0.00,1.00}{##1}}}
\expandafter\def\csname PY@tok@vc\endcsname{\def\PY@tc##1{\textcolor[rgb]{0.10,0.09,0.49}{##1}}}
\expandafter\def\csname PY@tok@vg\endcsname{\def\PY@tc##1{\textcolor[rgb]{0.10,0.09,0.49}{##1}}}
\expandafter\def\csname PY@tok@vi\endcsname{\def\PY@tc##1{\textcolor[rgb]{0.10,0.09,0.49}{##1}}}
\expandafter\def\csname PY@tok@vm\endcsname{\def\PY@tc##1{\textcolor[rgb]{0.10,0.09,0.49}{##1}}}
\expandafter\def\csname PY@tok@sa\endcsname{\def\PY@tc##1{\textcolor[rgb]{0.73,0.13,0.13}{##1}}}
\expandafter\def\csname PY@tok@sb\endcsname{\def\PY@tc##1{\textcolor[rgb]{0.73,0.13,0.13}{##1}}}
\expandafter\def\csname PY@tok@sc\endcsname{\def\PY@tc##1{\textcolor[rgb]{0.73,0.13,0.13}{##1}}}
\expandafter\def\csname PY@tok@dl\endcsname{\def\PY@tc##1{\textcolor[rgb]{0.73,0.13,0.13}{##1}}}
\expandafter\def\csname PY@tok@s2\endcsname{\def\PY@tc##1{\textcolor[rgb]{0.73,0.13,0.13}{##1}}}
\expandafter\def\csname PY@tok@sh\endcsname{\def\PY@tc##1{\textcolor[rgb]{0.73,0.13,0.13}{##1}}}
\expandafter\def\csname PY@tok@s1\endcsname{\def\PY@tc##1{\textcolor[rgb]{0.73,0.13,0.13}{##1}}}
\expandafter\def\csname PY@tok@mb\endcsname{\def\PY@tc##1{\textcolor[rgb]{0.40,0.40,0.40}{##1}}}
\expandafter\def\csname PY@tok@mf\endcsname{\def\PY@tc##1{\textcolor[rgb]{0.40,0.40,0.40}{##1}}}
\expandafter\def\csname PY@tok@mh\endcsname{\def\PY@tc##1{\textcolor[rgb]{0.40,0.40,0.40}{##1}}}
\expandafter\def\csname PY@tok@mi\endcsname{\def\PY@tc##1{\textcolor[rgb]{0.40,0.40,0.40}{##1}}}
\expandafter\def\csname PY@tok@il\endcsname{\def\PY@tc##1{\textcolor[rgb]{0.40,0.40,0.40}{##1}}}
\expandafter\def\csname PY@tok@mo\endcsname{\def\PY@tc##1{\textcolor[rgb]{0.40,0.40,0.40}{##1}}}
\expandafter\def\csname PY@tok@ch\endcsname{\let\PY@it=\textit\def\PY@tc##1{\textcolor[rgb]{0.25,0.50,0.50}{##1}}}
\expandafter\def\csname PY@tok@cm\endcsname{\let\PY@it=\textit\def\PY@tc##1{\textcolor[rgb]{0.25,0.50,0.50}{##1}}}
\expandafter\def\csname PY@tok@cpf\endcsname{\let\PY@it=\textit\def\PY@tc##1{\textcolor[rgb]{0.25,0.50,0.50}{##1}}}
\expandafter\def\csname PY@tok@c1\endcsname{\let\PY@it=\textit\def\PY@tc##1{\textcolor[rgb]{0.25,0.50,0.50}{##1}}}
\expandafter\def\csname PY@tok@cs\endcsname{\let\PY@it=\textit\def\PY@tc##1{\textcolor[rgb]{0.25,0.50,0.50}{##1}}}

\def\PYZbs{\char`\\}
\def\PYZus{\char`\_}
\def\PYZob{\char`\{}
\def\PYZcb{\char`\}}
\def\PYZca{\char`\^}
\def\PYZam{\char`\&}
\def\PYZlt{\char`\<}
\def\PYZgt{\char`\>}
\def\PYZsh{\char`\#}
\def\PYZpc{\char`\%}
\def\PYZdl{\char`\$}
\def\PYZhy{\char`\-}
\def\PYZsq{\char`\'}
\def\PYZdq{\char`\"}
\def\PYZti{\char`\~}
% for compatibility with earlier versions
\def\PYZat{@}
\def\PYZlb{[}
\def\PYZrb{]}
\makeatother


    % Exact colors from NB
    \definecolor{incolor}{rgb}{0.0, 0.0, 0.5}
    \definecolor{outcolor}{rgb}{0.545, 0.0, 0.0}




    
    % Prevent overflowing lines due to hard-to-break entities
    \sloppy 
    % Setup hyperref package
    \hypersetup{
      breaklinks=true,  % so long urls are correctly broken across lines
      colorlinks=true,
      urlcolor=urlcolor,
      linkcolor=linkcolor,
      citecolor=citecolor,
      }
    % Slightly bigger margins than the latex defaults
    
    %% margins.tplx %%

% margins
\textwidth=7in
\textheight=9.0in
\topmargin=-0.5in
\headheight=15pt
\headsep=.5in
\hoffset = -0.5in

\pagestyle{fancy}

    

    \begin{document}
    
    
    

    
    

    
    \hypertarget{lab-04---plotting-weather-data}{%
\section{Lab 04 - Plotting Weather
Data}\label{lab-04---plotting-weather-data}}

    \hypertarget{prelab}{%
\subsection{Prelab}\label{prelab}}

Before you start coding, read through this entire document, then create
a pseudocode outline of what your program will do. Pseudocode is read by
users, not read by the Python interpreter. Use separate the markdown
sections that describe what each of section of code will do. For
example, you could have a section for imports and a section to bring the
data into the Jupyter notebook. Describe what type of plots you will
build and, what the plots will look like in your pseudocode.

    \hypertarget{lab}{%
\subsection{Lab}\label{lab}}

Download the Excel spreadsheet file titled
\texttt{govt\_camp\_2013\_01\_to\_2016\_03.xlsx} and save it to your
preferred working directory. The directory where the Excel .xlsx file is
placed needs to be the same directory you build your Jupyter notebook
in. The Excel file contains daily weather data points gathered from the
USGS weather station in Government Camp, OR from Jan 1, 2009 to Mar 31,
2016. The weather measurements and their units are given in the column
headers (first row) of the Excel spreadsheet.

At the start of your lab, include a header in a Jupyter notebook
markdown cell that contains the lab name, your name, the course, quarter
and date. Below the header, import the libraries you will use in the
rest of the lab. Note that NumPy is imported as \texttt{np}, Pandas is
imported as \texttt{pd} and Matplotlib's pyplot library is imported as
\texttt{plt}. The three import aliases \texttt{np}, \texttt{pd}, and
\texttt{plt} are commonly used in Python. The line
\texttt{\%matplotlib\ inline} is a Jupyter notebook magic command, not
valid Python code. \texttt{\%matplotlib\ inline} results in Matplotlib
plots which are displayed directly in Jupyter notebook output cells,
instead of Matplotlib plots popping out as a separate windows. If you
build a \texttt{.py} file instead of a Jupyter notebook \texttt{.ipynb}
file, leave out \texttt{\%matplotlib\ inline}.

The Jupyter notebook code cell below shows the imports described above.

    \begin{Verbatim}[commandchars=\\\{\}]
{\color{incolor}In [{\color{incolor}1}]:} \PY{k+kn}{import} \PY{n+nn}{numpy} \PY{k}{as} \PY{n+nn}{np}
        \PY{k+kn}{import} \PY{n+nn}{pandas} \PY{k}{as} \PY{n+nn}{pd}
        \PY{k+kn}{import} \PY{n+nn}{matplotlib}\PY{n+nn}{.}\PY{n+nn}{pyplot} \PY{k}{as} \PY{n+nn}{plt}
        \PY{o}{\PYZpc{}}\PY{k}{matplotlib} inline
\end{Verbatim}


    Next, use Panda's \texttt{pd.read\_csv()} function to read the
\texttt{.xlsx} data file into a Pandas \emph{dataframe} called
\texttt{df}. The Panda's \texttt{df.head()} method shows the first
couple rows of the dataframe \texttt{df}.

The code below saves the Excel data file to the variable \texttt{df} and
shows the first couple rows of \texttt{df}.

    \begin{Verbatim}[commandchars=\\\{\}]
{\color{incolor}In [{\color{incolor}2}]:} \PY{n}{df} \PY{o}{=} \PY{n}{pd}\PY{o}{.}\PY{n}{read\PYZus{}excel}\PY{p}{(}\PY{l+s+s1}{\PYZsq{}}\PY{l+s+s1}{govt\PYZus{}camp\PYZus{}2013\PYZus{}01\PYZus{}to\PYZus{}2016\PYZus{}03.xlsx}\PY{l+s+s1}{\PYZsq{}}\PY{p}{)}
        \PY{n}{df}\PY{o}{.}\PY{n}{head}\PY{p}{(}\PY{p}{)}
\end{Verbatim}


\begin{Verbatim}[commandchars=\\\{\}]
{\color{outcolor}Out[{\color{outcolor}2}]:}              STATION           STATION\_NAME      DATE  \textbackslash{}
        0  GHCND:USC00353402  GOVERNMENT CAMP OR US  20130101   
        1  GHCND:USC00353402  GOVERNMENT CAMP OR US  20130102   
        2  GHCND:USC00353402  GOVERNMENT CAMP OR US  20130103   
        3  GHCND:USC00353402  GOVERNMENT CAMP OR US  20130104   
        4  GHCND:USC00353402  GOVERNMENT CAMP OR US  20130105   
        
           Precipitation\textbackslash{}n(inches)  Snow Depth\textbackslash{}n(inches)  Snowfall\textbackslash{}n(inches)  \textbackslash{}
        0                     0.00                    44                 0.0   
        1                     0.01                    42                 0.0   
        2                     0.00                    42                 0.0   
        3                     0.16                    41                 0.0   
        4                     0.00                    41                 0.0   
        
           Max Temp (°F)  Min Temp (°F)  
        0             29             12  
        1             27             11  
        2             20             12  
        3             31             14  
        4             37             28  
\end{Verbatim}
            
    The third column of the Pandas dataframe represents the date on which
the observations in that row were recorded. For example, the first row
of data (index 0) contains a value of \texttt{20130101} in the
\texttt{DATE} column. The value \texttt{20130101} means January 01,
2013. The date format is \texttt{YEAR-MONTH-DAY}.

Remove the first two columns of the dataframe \texttt{df} so that the
resulting first column in the dataframe is the \texttt{DATE} column. You
can use \texttt{df.head()} to view the dataframe after the two columns
are removed.

The code below drops the \texttt{STATION} and \texttt{STATION\_NAME}
columns from dataframe \texttt{df}.

    \begin{Verbatim}[commandchars=\\\{\}]
{\color{incolor}In [{\color{incolor}3}]:} \PY{n}{df}\PY{o}{.}\PY{n}{drop}\PY{p}{(}\PY{p}{[}\PY{l+s+s1}{\PYZsq{}}\PY{l+s+s1}{STATION}\PY{l+s+s1}{\PYZsq{}}\PY{p}{,} \PY{l+s+s1}{\PYZsq{}}\PY{l+s+s1}{STATION\PYZus{}NAME}\PY{l+s+s1}{\PYZsq{}}\PY{p}{]}\PY{p}{,} \PY{n}{axis}\PY{o}{=}\PY{l+m+mi}{1}\PY{p}{,} \PY{n}{inplace}\PY{o}{=}\PY{k+kc}{True}\PY{p}{)}
\end{Verbatim}


    In the resulting dataframe, first column now represents the date on
which the observations in that row were measured. For example, row 48 of
the dataframe contains a value of \texttt{20130219}. This means February
19, 2013. Replace column 1 of the dataframe \texttt{df} (the
\texttt{DATE} column) with 3 columns containing the \texttt{year} in the
first column, the \texttt{month} in the second column, and the
\texttt{day} in the third column. When you are finished adding (and
deleting the \texttt{DATE}) columns, your dataframe \texttt{df} should
be \texttt{1186\ x\ 8} and the dataframe's first row should have the
following values:

\begin{verbatim}
2013    1     1    0.00    44    0.0    29    12 
\end{verbatim}

The lines in the code cell below completes these operations. If you
rerun the code cell, make sure to start at the beginning of the
notebook.

    \begin{Verbatim}[commandchars=\\\{\}]
{\color{incolor}In [{\color{incolor}4}]:} \PY{n}{df}\PY{p}{[}\PY{l+s+s1}{\PYZsq{}}\PY{l+s+s1}{DATE}\PY{l+s+s1}{\PYZsq{}}\PY{p}{]} \PY{o}{=} \PY{n}{pd}\PY{o}{.}\PY{n}{to\PYZus{}datetime}\PY{p}{(}\PY{n}{df}\PY{p}{[}\PY{l+s+s1}{\PYZsq{}}\PY{l+s+s1}{DATE}\PY{l+s+s1}{\PYZsq{}}\PY{p}{]}\PY{p}{,} \PY{n+nb}{format}\PY{o}{=}\PY{l+s+s1}{\PYZsq{}}\PY{l+s+s1}{\PYZpc{}}\PY{l+s+s1}{Y}\PY{l+s+s1}{\PYZpc{}}\PY{l+s+s1}{m}\PY{l+s+si}{\PYZpc{}d}\PY{l+s+s1}{\PYZsq{}}\PY{p}{)}
        \PY{n}{df}\PY{o}{.}\PY{n}{insert}\PY{p}{(}\PY{l+m+mi}{0}\PY{p}{,} \PY{l+s+s1}{\PYZsq{}}\PY{l+s+s1}{year}\PY{l+s+s1}{\PYZsq{}}\PY{p}{,} \PY{n}{pd}\PY{o}{.}\PY{n}{DatetimeIndex}\PY{p}{(}\PY{n}{df}\PY{p}{[}\PY{l+s+s1}{\PYZsq{}}\PY{l+s+s1}{DATE}\PY{l+s+s1}{\PYZsq{}}\PY{p}{]}\PY{p}{)}\PY{o}{.}\PY{n}{year}\PY{p}{)}
        \PY{n}{df}\PY{o}{.}\PY{n}{insert}\PY{p}{(}\PY{l+m+mi}{1}\PY{p}{,} \PY{l+s+s1}{\PYZsq{}}\PY{l+s+s1}{month}\PY{l+s+s1}{\PYZsq{}}\PY{p}{,} \PY{n}{pd}\PY{o}{.}\PY{n}{DatetimeIndex}\PY{p}{(}\PY{n}{df}\PY{p}{[}\PY{l+s+s1}{\PYZsq{}}\PY{l+s+s1}{DATE}\PY{l+s+s1}{\PYZsq{}}\PY{p}{]}\PY{p}{)}\PY{o}{.}\PY{n}{month}\PY{p}{)}
        \PY{n}{df}\PY{o}{.}\PY{n}{insert}\PY{p}{(}\PY{l+m+mi}{2}\PY{p}{,} \PY{l+s+s1}{\PYZsq{}}\PY{l+s+s1}{day}\PY{l+s+s1}{\PYZsq{}}\PY{p}{,} \PY{n}{pd}\PY{o}{.}\PY{n}{DatetimeIndex}\PY{p}{(}\PY{n}{df}\PY{p}{[}\PY{l+s+s1}{\PYZsq{}}\PY{l+s+s1}{DATE}\PY{l+s+s1}{\PYZsq{}}\PY{p}{]}\PY{p}{)}\PY{o}{.}\PY{n}{day}\PY{p}{)}
        \PY{n}{df}\PY{o}{.}\PY{n}{drop}\PY{p}{(}\PY{p}{[}\PY{l+s+s1}{\PYZsq{}}\PY{l+s+s1}{DATE}\PY{l+s+s1}{\PYZsq{}}\PY{p}{]}\PY{p}{,} \PY{n}{axis}\PY{o}{=}\PY{l+m+mi}{1}\PY{p}{,} \PY{n}{inplace}\PY{o}{=}\PY{k+kc}{True}\PY{p}{)}
        \PY{n+nb}{print}\PY{p}{(}\PY{n}{df}\PY{o}{.}\PY{n}{shape}\PY{p}{)}
        \PY{n}{df}\PY{o}{.}\PY{n}{head}\PY{p}{(}\PY{p}{)}
\end{Verbatim}


    \begin{Verbatim}[commandchars=\\\{\}]
(1186, 8)

    \end{Verbatim}

\begin{Verbatim}[commandchars=\\\{\}]
{\color{outcolor}Out[{\color{outcolor}4}]:}    year  month  day  Precipitation\textbackslash{}n(inches)  Snow Depth\textbackslash{}n(inches)  \textbackslash{}
        0  2013      1    1                     0.00                    44   
        1  2013      1    2                     0.01                    42   
        2  2013      1    3                     0.00                    42   
        3  2013      1    4                     0.16                    41   
        4  2013      1    5                     0.00                    41   
        
           Snowfall\textbackslash{}n(inches)  Max Temp (°F)  Min Temp (°F)  
        0                 0.0             29             12  
        1                 0.0             27             11  
        2                 0.0             20             12  
        3                 0.0             31             14  
        4                 0.0             37             28  
\end{Verbatim}
            
    The last step of data cleaning is to convert the resulting dataframe
\texttt{df} into a NumPy array \texttt{wd}. Once the data is in a NumPy
array, you will be able to index and slice the data using regular NumPy
operations. Print the NumPy array with the command \texttt{print(wd)}
and see how the year, month and day have been converted to floats. Don't
forget what the columns correspond to.

The code cell below shows how to convert the dataframe \texttt{df} into
a NumPy array called \texttt{wd}.

    \begin{Verbatim}[commandchars=\\\{\}]
{\color{incolor}In [{\color{incolor}5}]:} \PY{n}{wd} \PY{o}{=} \PY{n}{np}\PY{o}{.}\PY{n}{array}\PY{p}{(}\PY{n}{df}\PY{p}{)}
        \PY{n+nb}{print}\PY{p}{(}\PY{n}{wd}\PY{p}{)}
\end{Verbatim}


    \begin{Verbatim}[commandchars=\\\{\}]
[[2.013e+03 1.000e+00 1.000e+00 {\ldots} 0.000e+00 2.900e+01 1.200e+01]
 [2.013e+03 1.000e+00 2.000e+00 {\ldots} 0.000e+00 2.700e+01 1.100e+01]
 [2.013e+03 1.000e+00 3.000e+00 {\ldots} 0.000e+00 2.000e+01 1.200e+01]
 {\ldots}
 [2.016e+03 3.000e+00 2.900e+01 {\ldots} 0.000e+00 3.300e+01 2.700e+01]
 [2.016e+03 3.000e+00 3.000e+01 {\ldots} 0.000e+00 4.700e+01 2.700e+01]
 [2.016e+03 3.000e+00 3.100e+01 {\ldots} 0.000e+00 5.900e+01 3.200e+01]]

    \end{Verbatim}
\newpage
    \hypertarget{create-a-line-plot}{%
\subsubsection{Create a line plot}\label{create-a-line-plot}}

Once the data is cleaned up, create a line plot with Matplotlib of
precipitation in inches vs.~time (in days since Jan 01, 2013). Include
an x-axis label, a y-axis label, and a title on your plot. Look back at
the column headers in the dataframe \texttt{df} to determine which
column corresponds to \texttt{Precipitation\ (inches)}. Use indexing to
pull out all of the values from the precipitation column to build your
plot.

The resulting plot should look something like the plot below:

\begin{figure}[h!]
\centering
\includegraphics{images/plot1.png}
\caption{Plot 1: Line Plot}
\end{figure}
\newpage
    \hypertarget{create-a-line-plot-with-two-lines}{%
\subsubsection{Create a line plot with two
lines}\label{create-a-line-plot-with-two-lines}}

Next, create a Matplotlib line plot that shows the maximum and minimum
temperatures in 2013 on two lines. Look back at the datafrome
\texttt{df} to determine which columns correspond to the maximum and
minimum temperatures recorded each day. The maximum temp should be a red
line and the minimum temp should be a blue line. Include a legend that
shows \texttt{Max\ Temp} and \texttt{Min\ Temp}. Include axis labels and
a title on the plot.

The resulting plot should look something like the plot below:

\begin{figure}[h!]
\centering
\includegraphics{images/plot2.png}
\caption{Plot 2: Multi-line Plot}
\end{figure}

    \hypertarget{create-4-subplots}{%
\subsubsection{Create 4 subplots}\label{create-4-subplots}}

The next task is to create a Matplotlib figure containing four subplots
that looks like the figure with four plots shown below. The four
subplots should be stacked on top of each other. Each subplot should
contain the precipitation data vs.~time (day of the year) in one year.
The first subplot shows precipitation in 2013, the second subplot shows
precipitation in 2014, etc. Include a title on each subplot and a y-axis
label on each subplot. On the bottom subplot (Precipitation in 2016),
include an x-axis label.

Use NumPy's \texttt{np.where()} function or boolean masks to save the
precipitation data (number of inches only) exclusively for days in the
year 2013 to a NumPy array called \texttt{precip\_2013}. Use the
\texttt{plt.subplots()} method to plot the \texttt{precip\_2013} data in
the first subplot.

Complete the same type of indexing operation on the precipitation data
for 2014, then 2015, then 2016. Create a subplot for each year. The plot
for 2016 should include an x-axis label and have the same axis limits as
the x-axis limits in the other subplots.

\begin{quote}
Hint: When \texttt{plt.subplots()} is called, include the keyword
argument \texttt{figsize=(9,9)}. This keyword argument makes the figure
larger and leaves more space for the axis labels and titles of each
subplot. Modify the numbers \texttt{(9,9)} to change the figure width
and height.
\end{quote}

The resulting figure which contains four subplots, one subplot for each
year, should look something like below. The command
\texttt{plt.tight\_layout()} can also help with subplot spacing.

\begin{figure}[h!]
\centering
\includegraphics{images/plot3.png}
\caption{Plot 3: Four Subplots}
\end{figure}

\newpage
    \hypertarget{y-y-plot}{%
\subsubsection{y-y plot}\label{y-y-plot}}

The next plot you will build is a y-y plot. A y-y plot has two different
y-axes. Each set or sets of data corresponds to one of the separate
y-axes. Y-y plots are useful when two types of data contain the same
x-values (like time) but very different scales of y-values (like snow
fall and snow depth).

Use boolean masking to save some of the \textbf{snow depth} data into a
NumPy array called \texttt{winter13\_14\_sd}. Remember which column
headers correspond to which measurement in the dataframe \texttt{df}.
Refer to the dataframe or Excel file, if need be, to see which column
corresponds snow depth. Your snow depth data set should only include
measurements from the first day of November 2013 to the last day of
February 2014. Slicing or indexing out snow depth data in this date
range is probably be the most difficult part of the lab.

\begin{quote}
Hint: Create multiple boolean masks and then combine the boolean masks
with NumPy's \texttt{np.logical\_and()} and \texttt{np.logical\_or()}
functions. These two NumPy functions operate like Python's \texttt{and},
\texttt{or} keywords, but can be applied to arrays of boolean values.
\end{quote}

Index out the same days' (Nov 1, 2013 to Feb 27, 2014) \textbf{snow
fall} data and save the snow fall data in this date range to a NumPy
array called \texttt{winter13\_14\_sf}.

Use Matplotlib's \texttt{ax2\ =\ ax1.twinx()} method to plot both sets
of data on the same plot (same x-axis), but with different y-axes.

Notice the figure has a different vertical axis on the left compared to
the vertical axis on the right. Let Matplotlib choose your axis limits
and line colors. Be sure to include the grid lines, x-axis label, y-axis
labels, title, and legend as seen in the figure below. It is tricky to
include both entries in the legend. Do your best, but don't waste too
much time configuring the legend.

\begin{figure}[h!]
\centering
\includegraphics{images/yyplot.png}
\caption{Plot 4: y-y Plot}
\end{figure}
\newpage
    \hypertarget{histogram}{%
\subsubsection{Histogram}\label{histogram}}

The final plot to create in this lab is a histogram. A histogram is a
statistical plot that shows the number of values that fall into
different value ranges, called bins. Use your \texttt{winter13\_14\_sf}
NumPy array to build the histogram. Include 10 bins in the histogram.
Make sure to match all of the labels and title from the example
histogram below.

\begin{figure}[h!]
\centering
\includegraphics{images/plot5.png}
\caption{Plot 5: Histogram}
\end{figure}

    \hypertarget{deliverables}{%
\subsection{Deliverables}\label{deliverables}}

Each student's submission for the lab must be a single Jupyter Notebook.
The Jupyter Notebook must contain markdown cells that explain the code,
code cells that contain the code, and output cells that contain a total
of five plots. The five plots produced by the code cells in the notebook
should be:

\begin{itemize}
\item
  A line plot
\item
  A line plot with two lines and a legend
\item
  A figure that contains four sub-plots that shows precipitation verses
  time in four different years
\item
  A y-y plot that shows snow level on one y-axis and snow depth on the
  other y-axis
\item
  A histogram of snow fall
\end{itemize}

Make sure to run all the cells in your Jupyter notebook before
submitting the file. Upload your \textbf{\emph{lab4.ipynb}} file to the
Lab 4 drop box in D2L.

    \hypertarget{by-d.-kruger-modified-by-p.-kazarinoff-portland-community-college-2019}{%
\paragraph{\texorpdfstring{\emph{By D. Kruger, modified by P.
Kazarinoff, Portland Community College,
2019}}{By D. Kruger, modified by P. Kazarinoff, Portland Community College, 2019}}\label{by-d.-kruger-modified-by-p.-kazarinoff-portland-community-college-2019}}


    % Add a bibliography block to the postdoc
    
    
    
    \end{document}

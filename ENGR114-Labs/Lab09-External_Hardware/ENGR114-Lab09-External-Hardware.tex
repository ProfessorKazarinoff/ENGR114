%% ENGR114_lab_assignment.tplx %%
%
% Built off of the article.tplx template %


% Default to the notebook output style

    


% Inherit from the specified cell style.




    
    \documentclass[11pt]{article}

    
    
    %% installed packages_rev2.tplx %%

\usepackage{fancyhdr}
\usepackage{lastpage}
\usepackage{framed,color}
\definecolor{shadecolor}{rgb}{.8,.8,.8}
\usepackage{titlesec}
% no indent on any paragraphs, vertical spacing between paragraphs is set to 1em
\usepackage[]{parskip}  % add [skip=1em] if the compiler will allow.

% for MATLAB syntax highlighting
\usepackage{listings}             % Include the listings-package
\definecolor{mygray}{rgb}{0.8,0.8,0.8} % color values Red, Green, Blue
\definecolor{mygreen}{RGB}{28,172,0}
\definecolor{mylilas}{RGB}{170,55,241}
    
    \usepackage[T1]{fontenc}
    % Nicer default font (+ math font) than Computer Modern for most use cases
    \usepackage{mathpazo}

    % Basic figure setup, for now with no caption control since it's done
    % automatically by Pandoc (which extracts ![](path) syntax from Markdown).
    \usepackage{graphicx}
    % We will generate all images so they have a width \maxwidth. This means
    % that they will get their normal width if they fit onto the page, but
    % are scaled down if they would overflow the margins.
    \makeatletter
    \def\maxwidth{\ifdim\Gin@nat@width>\linewidth\linewidth
    \else\Gin@nat@width\fi}
    \makeatother
    \let\Oldincludegraphics\includegraphics
    % Set max figure width to be 80% of text width, for now hardcoded.
    \renewcommand{\includegraphics}[1]{\Oldincludegraphics[width=.8\maxwidth]{#1}}
    % Ensure that by default, figures have no caption (until we provide a
    % proper Figure object with a Caption API and a way to capture that
    % in the conversion process - todo).
    \usepackage{caption}
    \DeclareCaptionLabelFormat{nolabel}{}
    \captionsetup{labelformat=nolabel}

    \usepackage{adjustbox} % Used to constrain images to a maximum size 
    \usepackage{xcolor} % Allow colors to be defined
    \usepackage{enumerate} % Needed for markdown enumerations to work
    \usepackage{geometry} % Used to adjust the document margins
    \usepackage{amsmath} % Equations
    \usepackage{amssymb} % Equations
    \usepackage{textcomp} % defines textquotesingle
    % Hack from http://tex.stackexchange.com/a/47451/13684:
    \AtBeginDocument{%
        \def\PYZsq{\textquotesingle}% Upright quotes in Pygmentized code
    }
    \usepackage{upquote} % Upright quotes for verbatim code
    \usepackage{eurosym} % defines \euro
    \usepackage[mathletters]{ucs} % Extended unicode (utf-8) support
    \usepackage[utf8x]{inputenc} % Allow utf-8 characters in the tex document
    \usepackage{fancyvrb} % verbatim replacement that allows latex
    \usepackage{grffile} % extends the file name processing of package graphics 
                         % to support a larger range 
    % The hyperref package gives us a pdf with properly built
    % internal navigation ('pdf bookmarks' for the table of contents,
    % internal cross-reference links, web links for URLs, etc.)
    \usepackage{hyperref}
    \usepackage{longtable} % longtable support required by pandoc >1.10
    \usepackage{booktabs}  % table support for pandoc > 1.12.2
    \usepackage[inline]{enumitem} % IRkernel/repr support (it uses the enumerate* environment)
    \usepackage[normalem]{ulem} % ulem is needed to support strikethroughs (\sout)
                                % normalem makes italics be italics, not underlines
    


    
    %% lab_title.tplx %% 
 
\newcommand{\labtitle}{Lab09 External Hardware} 
    %% header_and_footer.tplx %%

% Header and Footer
\lhead{\textbf{\labtitle}}
\rhead{ENGR114 Engineering Programming}
\lfoot{Portland Community College, \the\year}
\cfoot{}
\rfoot{\thepage~of~\pageref{LastPage}}  % must compile twice for LastPage

%lines below header and above footer
\renewcommand{\headrulewidth}{0.4pt}
\renewcommand{\footrulewidth}{0.4pt}

% Tabs
\newcommand{\itab}[1]{\hspace{0em}\rlap{#1}}
\newcommand{\tab}[1]{\hspace{.4\textwidth}\rlap{#1}}
\newcommand{\tabA}[1]{\hspace{.2\textwidth}\rlap{#1}}
    %% title_sec_formatting.tplx %%

\titleformat{\section}[block]{\LARGE\bfseries\filcenter}{}{1em}{}

\titleformat{\subsection}[hang]{\Large\bfseries}{}{1em}{}
\titlespacing{\subsection}{-1.4em}{1.5em}{1em}

\titleformat{\subsubsection}[hang]{\large\bfseries}{}{1em}{}
\titlespacing{\subsubsection}{-1.1em}{1.5em}{0.8em}
    
        \title{Problem Solving 101 with Python}
        \author{Peter D. Kazarinoff, PhD}
        \date{}
    
    
    
    % Colors for the hyperref package
    \definecolor{urlcolor}{rgb}{0,.145,.698}
    \definecolor{linkcolor}{rgb}{.71,0.21,0.01}
    \definecolor{citecolor}{rgb}{.12,.54,.11}

    % ANSI colors
    \definecolor{ansi-black}{HTML}{3E424D}
    \definecolor{ansi-black-intense}{HTML}{282C36}
    \definecolor{ansi-red}{HTML}{E75C58}
    \definecolor{ansi-red-intense}{HTML}{B22B31}
    \definecolor{ansi-green}{HTML}{00A250}
    \definecolor{ansi-green-intense}{HTML}{007427}
    \definecolor{ansi-yellow}{HTML}{DDB62B}
    \definecolor{ansi-yellow-intense}{HTML}{B27D12}
    \definecolor{ansi-blue}{HTML}{208FFB}
    \definecolor{ansi-blue-intense}{HTML}{0065CA}
    \definecolor{ansi-magenta}{HTML}{D160C4}
    \definecolor{ansi-magenta-intense}{HTML}{A03196}
    \definecolor{ansi-cyan}{HTML}{60C6C8}
    \definecolor{ansi-cyan-intense}{HTML}{258F8F}
    \definecolor{ansi-white}{HTML}{C5C1B4}
    \definecolor{ansi-white-intense}{HTML}{A1A6B2}

    % commands and environments needed by pandoc snippets
    % extracted from the output of `pandoc -s`
    \providecommand{\tightlist}{%
      \setlength{\itemsep}{0pt}\setlength{\parskip}{0pt}}
    \DefineVerbatimEnvironment{Highlighting}{Verbatim}{commandchars=\\\{\}}
    % Add ',fontsize=\small' for more characters per line
    \newenvironment{Shaded}{}{}
    \newcommand{\KeywordTok}[1]{\textcolor[rgb]{0.00,0.44,0.13}{\textbf{{#1}}}}
    \newcommand{\DataTypeTok}[1]{\textcolor[rgb]{0.56,0.13,0.00}{{#1}}}
    \newcommand{\DecValTok}[1]{\textcolor[rgb]{0.25,0.63,0.44}{{#1}}}
    \newcommand{\BaseNTok}[1]{\textcolor[rgb]{0.25,0.63,0.44}{{#1}}}
    \newcommand{\FloatTok}[1]{\textcolor[rgb]{0.25,0.63,0.44}{{#1}}}
    \newcommand{\CharTok}[1]{\textcolor[rgb]{0.25,0.44,0.63}{{#1}}}
    \newcommand{\StringTok}[1]{\textcolor[rgb]{0.25,0.44,0.63}{{#1}}}
    \newcommand{\CommentTok}[1]{\textcolor[rgb]{0.38,0.63,0.69}{\textit{{#1}}}}
    \newcommand{\OtherTok}[1]{\textcolor[rgb]{0.00,0.44,0.13}{{#1}}}
    \newcommand{\AlertTok}[1]{\textcolor[rgb]{1.00,0.00,0.00}{\textbf{{#1}}}}
    \newcommand{\FunctionTok}[1]{\textcolor[rgb]{0.02,0.16,0.49}{{#1}}}
    \newcommand{\RegionMarkerTok}[1]{{#1}}
    \newcommand{\ErrorTok}[1]{\textcolor[rgb]{1.00,0.00,0.00}{\textbf{{#1}}}}
    \newcommand{\NormalTok}[1]{{#1}}
    
    % Additional commands for more recent versions of Pandoc
    \newcommand{\ConstantTok}[1]{\textcolor[rgb]{0.53,0.00,0.00}{{#1}}}
    \newcommand{\SpecialCharTok}[1]{\textcolor[rgb]{0.25,0.44,0.63}{{#1}}}
    \newcommand{\VerbatimStringTok}[1]{\textcolor[rgb]{0.25,0.44,0.63}{{#1}}}
    \newcommand{\SpecialStringTok}[1]{\textcolor[rgb]{0.73,0.40,0.53}{{#1}}}
    \newcommand{\ImportTok}[1]{{#1}}
    \newcommand{\DocumentationTok}[1]{\textcolor[rgb]{0.73,0.13,0.13}{\textit{{#1}}}}
    \newcommand{\AnnotationTok}[1]{\textcolor[rgb]{0.38,0.63,0.69}{\textbf{\textit{{#1}}}}}
    \newcommand{\CommentVarTok}[1]{\textcolor[rgb]{0.38,0.63,0.69}{\textbf{\textit{{#1}}}}}
    \newcommand{\VariableTok}[1]{\textcolor[rgb]{0.10,0.09,0.49}{{#1}}}
    \newcommand{\ControlFlowTok}[1]{\textcolor[rgb]{0.00,0.44,0.13}{\textbf{{#1}}}}
    \newcommand{\OperatorTok}[1]{\textcolor[rgb]{0.40,0.40,0.40}{{#1}}}
    \newcommand{\BuiltInTok}[1]{{#1}}
    \newcommand{\ExtensionTok}[1]{{#1}}
    \newcommand{\PreprocessorTok}[1]{\textcolor[rgb]{0.74,0.48,0.00}{{#1}}}
    \newcommand{\AttributeTok}[1]{\textcolor[rgb]{0.49,0.56,0.16}{{#1}}}
    \newcommand{\InformationTok}[1]{\textcolor[rgb]{0.38,0.63,0.69}{\textbf{\textit{{#1}}}}}
    \newcommand{\WarningTok}[1]{\textcolor[rgb]{0.38,0.63,0.69}{\textbf{\textit{{#1}}}}}
    
    
    % Define a nice break command that doesn't care if a line doesn't already
    % exist.
    \def\br{\hspace*{\fill} \\* }
    % Math Jax compatability definitions
    \def\gt{>}
    \def\lt{<}
    % Document parameters
    
        \title{Problem Solving 101 with Python}
        \author{Peter D. Kazarinoff, PhD}
        \date{}
    
    
    
    

    % Pygments definitions
    
\makeatletter
\def\PY@reset{\let\PY@it=\relax \let\PY@bf=\relax%
    \let\PY@ul=\relax \let\PY@tc=\relax%
    \let\PY@bc=\relax \let\PY@ff=\relax}
\def\PY@tok#1{\csname PY@tok@#1\endcsname}
\def\PY@toks#1+{\ifx\relax#1\empty\else%
    \PY@tok{#1}\expandafter\PY@toks\fi}
\def\PY@do#1{\PY@bc{\PY@tc{\PY@ul{%
    \PY@it{\PY@bf{\PY@ff{#1}}}}}}}
\def\PY#1#2{\PY@reset\PY@toks#1+\relax+\PY@do{#2}}

\expandafter\def\csname PY@tok@w\endcsname{\def\PY@tc##1{\textcolor[rgb]{0.73,0.73,0.73}{##1}}}
\expandafter\def\csname PY@tok@c\endcsname{\let\PY@it=\textit\def\PY@tc##1{\textcolor[rgb]{0.25,0.50,0.50}{##1}}}
\expandafter\def\csname PY@tok@cp\endcsname{\def\PY@tc##1{\textcolor[rgb]{0.74,0.48,0.00}{##1}}}
\expandafter\def\csname PY@tok@k\endcsname{\let\PY@bf=\textbf\def\PY@tc##1{\textcolor[rgb]{0.00,0.50,0.00}{##1}}}
\expandafter\def\csname PY@tok@kp\endcsname{\def\PY@tc##1{\textcolor[rgb]{0.00,0.50,0.00}{##1}}}
\expandafter\def\csname PY@tok@kt\endcsname{\def\PY@tc##1{\textcolor[rgb]{0.69,0.00,0.25}{##1}}}
\expandafter\def\csname PY@tok@o\endcsname{\def\PY@tc##1{\textcolor[rgb]{0.40,0.40,0.40}{##1}}}
\expandafter\def\csname PY@tok@ow\endcsname{\let\PY@bf=\textbf\def\PY@tc##1{\textcolor[rgb]{0.67,0.13,1.00}{##1}}}
\expandafter\def\csname PY@tok@nb\endcsname{\def\PY@tc##1{\textcolor[rgb]{0.00,0.50,0.00}{##1}}}
\expandafter\def\csname PY@tok@nf\endcsname{\def\PY@tc##1{\textcolor[rgb]{0.00,0.00,1.00}{##1}}}
\expandafter\def\csname PY@tok@nc\endcsname{\let\PY@bf=\textbf\def\PY@tc##1{\textcolor[rgb]{0.00,0.00,1.00}{##1}}}
\expandafter\def\csname PY@tok@nn\endcsname{\let\PY@bf=\textbf\def\PY@tc##1{\textcolor[rgb]{0.00,0.00,1.00}{##1}}}
\expandafter\def\csname PY@tok@ne\endcsname{\let\PY@bf=\textbf\def\PY@tc##1{\textcolor[rgb]{0.82,0.25,0.23}{##1}}}
\expandafter\def\csname PY@tok@nv\endcsname{\def\PY@tc##1{\textcolor[rgb]{0.10,0.09,0.49}{##1}}}
\expandafter\def\csname PY@tok@no\endcsname{\def\PY@tc##1{\textcolor[rgb]{0.53,0.00,0.00}{##1}}}
\expandafter\def\csname PY@tok@nl\endcsname{\def\PY@tc##1{\textcolor[rgb]{0.63,0.63,0.00}{##1}}}
\expandafter\def\csname PY@tok@ni\endcsname{\let\PY@bf=\textbf\def\PY@tc##1{\textcolor[rgb]{0.60,0.60,0.60}{##1}}}
\expandafter\def\csname PY@tok@na\endcsname{\def\PY@tc##1{\textcolor[rgb]{0.49,0.56,0.16}{##1}}}
\expandafter\def\csname PY@tok@nt\endcsname{\let\PY@bf=\textbf\def\PY@tc##1{\textcolor[rgb]{0.00,0.50,0.00}{##1}}}
\expandafter\def\csname PY@tok@nd\endcsname{\def\PY@tc##1{\textcolor[rgb]{0.67,0.13,1.00}{##1}}}
\expandafter\def\csname PY@tok@s\endcsname{\def\PY@tc##1{\textcolor[rgb]{0.73,0.13,0.13}{##1}}}
\expandafter\def\csname PY@tok@sd\endcsname{\let\PY@it=\textit\def\PY@tc##1{\textcolor[rgb]{0.73,0.13,0.13}{##1}}}
\expandafter\def\csname PY@tok@si\endcsname{\let\PY@bf=\textbf\def\PY@tc##1{\textcolor[rgb]{0.73,0.40,0.53}{##1}}}
\expandafter\def\csname PY@tok@se\endcsname{\let\PY@bf=\textbf\def\PY@tc##1{\textcolor[rgb]{0.73,0.40,0.13}{##1}}}
\expandafter\def\csname PY@tok@sr\endcsname{\def\PY@tc##1{\textcolor[rgb]{0.73,0.40,0.53}{##1}}}
\expandafter\def\csname PY@tok@ss\endcsname{\def\PY@tc##1{\textcolor[rgb]{0.10,0.09,0.49}{##1}}}
\expandafter\def\csname PY@tok@sx\endcsname{\def\PY@tc##1{\textcolor[rgb]{0.00,0.50,0.00}{##1}}}
\expandafter\def\csname PY@tok@m\endcsname{\def\PY@tc##1{\textcolor[rgb]{0.40,0.40,0.40}{##1}}}
\expandafter\def\csname PY@tok@gh\endcsname{\let\PY@bf=\textbf\def\PY@tc##1{\textcolor[rgb]{0.00,0.00,0.50}{##1}}}
\expandafter\def\csname PY@tok@gu\endcsname{\let\PY@bf=\textbf\def\PY@tc##1{\textcolor[rgb]{0.50,0.00,0.50}{##1}}}
\expandafter\def\csname PY@tok@gd\endcsname{\def\PY@tc##1{\textcolor[rgb]{0.63,0.00,0.00}{##1}}}
\expandafter\def\csname PY@tok@gi\endcsname{\def\PY@tc##1{\textcolor[rgb]{0.00,0.63,0.00}{##1}}}
\expandafter\def\csname PY@tok@gr\endcsname{\def\PY@tc##1{\textcolor[rgb]{1.00,0.00,0.00}{##1}}}
\expandafter\def\csname PY@tok@ge\endcsname{\let\PY@it=\textit}
\expandafter\def\csname PY@tok@gs\endcsname{\let\PY@bf=\textbf}
\expandafter\def\csname PY@tok@gp\endcsname{\let\PY@bf=\textbf\def\PY@tc##1{\textcolor[rgb]{0.00,0.00,0.50}{##1}}}
\expandafter\def\csname PY@tok@go\endcsname{\def\PY@tc##1{\textcolor[rgb]{0.53,0.53,0.53}{##1}}}
\expandafter\def\csname PY@tok@gt\endcsname{\def\PY@tc##1{\textcolor[rgb]{0.00,0.27,0.87}{##1}}}
\expandafter\def\csname PY@tok@err\endcsname{\def\PY@bc##1{\setlength{\fboxsep}{0pt}\fcolorbox[rgb]{1.00,0.00,0.00}{1,1,1}{\strut ##1}}}
\expandafter\def\csname PY@tok@kc\endcsname{\let\PY@bf=\textbf\def\PY@tc##1{\textcolor[rgb]{0.00,0.50,0.00}{##1}}}
\expandafter\def\csname PY@tok@kd\endcsname{\let\PY@bf=\textbf\def\PY@tc##1{\textcolor[rgb]{0.00,0.50,0.00}{##1}}}
\expandafter\def\csname PY@tok@kn\endcsname{\let\PY@bf=\textbf\def\PY@tc##1{\textcolor[rgb]{0.00,0.50,0.00}{##1}}}
\expandafter\def\csname PY@tok@kr\endcsname{\let\PY@bf=\textbf\def\PY@tc##1{\textcolor[rgb]{0.00,0.50,0.00}{##1}}}
\expandafter\def\csname PY@tok@bp\endcsname{\def\PY@tc##1{\textcolor[rgb]{0.00,0.50,0.00}{##1}}}
\expandafter\def\csname PY@tok@fm\endcsname{\def\PY@tc##1{\textcolor[rgb]{0.00,0.00,1.00}{##1}}}
\expandafter\def\csname PY@tok@vc\endcsname{\def\PY@tc##1{\textcolor[rgb]{0.10,0.09,0.49}{##1}}}
\expandafter\def\csname PY@tok@vg\endcsname{\def\PY@tc##1{\textcolor[rgb]{0.10,0.09,0.49}{##1}}}
\expandafter\def\csname PY@tok@vi\endcsname{\def\PY@tc##1{\textcolor[rgb]{0.10,0.09,0.49}{##1}}}
\expandafter\def\csname PY@tok@vm\endcsname{\def\PY@tc##1{\textcolor[rgb]{0.10,0.09,0.49}{##1}}}
\expandafter\def\csname PY@tok@sa\endcsname{\def\PY@tc##1{\textcolor[rgb]{0.73,0.13,0.13}{##1}}}
\expandafter\def\csname PY@tok@sb\endcsname{\def\PY@tc##1{\textcolor[rgb]{0.73,0.13,0.13}{##1}}}
\expandafter\def\csname PY@tok@sc\endcsname{\def\PY@tc##1{\textcolor[rgb]{0.73,0.13,0.13}{##1}}}
\expandafter\def\csname PY@tok@dl\endcsname{\def\PY@tc##1{\textcolor[rgb]{0.73,0.13,0.13}{##1}}}
\expandafter\def\csname PY@tok@s2\endcsname{\def\PY@tc##1{\textcolor[rgb]{0.73,0.13,0.13}{##1}}}
\expandafter\def\csname PY@tok@sh\endcsname{\def\PY@tc##1{\textcolor[rgb]{0.73,0.13,0.13}{##1}}}
\expandafter\def\csname PY@tok@s1\endcsname{\def\PY@tc##1{\textcolor[rgb]{0.73,0.13,0.13}{##1}}}
\expandafter\def\csname PY@tok@mb\endcsname{\def\PY@tc##1{\textcolor[rgb]{0.40,0.40,0.40}{##1}}}
\expandafter\def\csname PY@tok@mf\endcsname{\def\PY@tc##1{\textcolor[rgb]{0.40,0.40,0.40}{##1}}}
\expandafter\def\csname PY@tok@mh\endcsname{\def\PY@tc##1{\textcolor[rgb]{0.40,0.40,0.40}{##1}}}
\expandafter\def\csname PY@tok@mi\endcsname{\def\PY@tc##1{\textcolor[rgb]{0.40,0.40,0.40}{##1}}}
\expandafter\def\csname PY@tok@il\endcsname{\def\PY@tc##1{\textcolor[rgb]{0.40,0.40,0.40}{##1}}}
\expandafter\def\csname PY@tok@mo\endcsname{\def\PY@tc##1{\textcolor[rgb]{0.40,0.40,0.40}{##1}}}
\expandafter\def\csname PY@tok@ch\endcsname{\let\PY@it=\textit\def\PY@tc##1{\textcolor[rgb]{0.25,0.50,0.50}{##1}}}
\expandafter\def\csname PY@tok@cm\endcsname{\let\PY@it=\textit\def\PY@tc##1{\textcolor[rgb]{0.25,0.50,0.50}{##1}}}
\expandafter\def\csname PY@tok@cpf\endcsname{\let\PY@it=\textit\def\PY@tc##1{\textcolor[rgb]{0.25,0.50,0.50}{##1}}}
\expandafter\def\csname PY@tok@c1\endcsname{\let\PY@it=\textit\def\PY@tc##1{\textcolor[rgb]{0.25,0.50,0.50}{##1}}}
\expandafter\def\csname PY@tok@cs\endcsname{\let\PY@it=\textit\def\PY@tc##1{\textcolor[rgb]{0.25,0.50,0.50}{##1}}}

\def\PYZbs{\char`\\}
\def\PYZus{\char`\_}
\def\PYZob{\char`\{}
\def\PYZcb{\char`\}}
\def\PYZca{\char`\^}
\def\PYZam{\char`\&}
\def\PYZlt{\char`\<}
\def\PYZgt{\char`\>}
\def\PYZsh{\char`\#}
\def\PYZpc{\char`\%}
\def\PYZdl{\char`\$}
\def\PYZhy{\char`\-}
\def\PYZsq{\char`\'}
\def\PYZdq{\char`\"}
\def\PYZti{\char`\~}
% for compatibility with earlier versions
\def\PYZat{@}
\def\PYZlb{[}
\def\PYZrb{]}
\makeatother


    % Exact colors from NB
    \definecolor{incolor}{rgb}{0.0, 0.0, 0.5}
    \definecolor{outcolor}{rgb}{0.545, 0.0, 0.0}




    
    % Prevent overflowing lines due to hard-to-break entities
    \sloppy 
    % Setup hyperref package
    \hypersetup{
      breaklinks=true,  % so long urls are correctly broken across lines
      colorlinks=true,
      urlcolor=urlcolor,
      linkcolor=linkcolor,
      citecolor=citecolor,
      }
    % Slightly bigger margins than the latex defaults
    
    %% margins.tplx %%

% margins
\textwidth=7in
\textheight=9.0in
\topmargin=-0.5in
\headheight=15pt
\headsep=.5in
\hoffset = -0.5in

\pagestyle{fancy}

    

    \begin{document}
    
    
    

    
    

    
    \hypertarget{lab-09---external-hardware}{%
\section{Lab 09 - External Hardware}\label{lab-09---external-hardware}}

    {[}image of Arduino and computer{]}

    \hypertarget{prelab}{%
\subsection{Prelab}\label{prelab}}

Before starting the lab, familiarize yourself with what an Arduino is
and what an Arduino can be used for. You will find numerous projects if
you Google ``Arduino project''. Try to download the Arduino IDE using
the following link:

\begin{quote}
https://www.arduino.cc/en/Main/Software
\end{quote}

Be sure to select: {[}Windows ZIP for non-admin install{]} as new
software can not be installed on lab computers without administrator
privileges.

\begin{figure}
\centering
\includegraphics{images/arduino_download_page.png}
\caption{Arduino IDE Download Page}
\end{figure}

Investigate the Arudino IDE (launch by double clicking Arduino.exe). See
what is available in the {[}File{]} --\textgreater{} {[}Examples{]}
menu. Before lab, also familiarize yourself with the concept of serial
communication. In this lab, Python will communicate with an Arduino over
a serial connection. Serial communication is one of the older computer
hardware communication specifications and a precursor to USB (universal
\emph{serial} bus) communication used by keyboards, mice, printers,
thumb drives etc.

Part of this lab will follow a tutorial from the Sparkfun Inventor's Kit
(https://www.sparkfun.com/products/retired/12060) click on the
{[}Documents{]} link on the product page to review the guide.

    \hypertarget{lab}{%
\subsection{Lab}\label{lab}}

For this group lab assignment, your group will use Python to interact
with an Arduino to dynamically turn on and off an LED, and then use
Python to collect and plot the sensor data. Your group will construct
two Python scripts and use tow Arduino scripts to complete these tasks.
At the end of the lab, you will be able to use Python to interact with
external hardware.

    An outline of the steps to complete this lab is below:

\hypertarget{part-1-turn-an-led-on-and-off-with-python-and-an-arduino}{%
\subsubsection{Part 1: Turn an LED on and off with Python and an
Arduino}\label{part-1-turn-an-led-on-and-off-with-python-and-an-arduino}}

\begin{enumerate}
\def\labelenumi{(\alph{enumi})}
\item
  Download the Arduino IDE
\item
  Wire an LED and resistor to the Arduino
\item
  Upload the Arduino example sketch blink.ino onto the Arduino. Confirm
  your Arduino and LED blinks.
\item
  Load the Arduino example sketch PhysicalPixel.ino
\item
  Use the serial monitor to turn the Arduino LED on and off
\item
  Build a Python script to turn the Arduino LED on and off
\end{enumerate}

    \hypertarget{part-2-measure-sensor-output-with-python-and-an-arduino}{%
\subsubsection{Part 2: Measure sensor output with Python and an
Arduino}\label{part-2-measure-sensor-output-with-python-and-an-arduino}}

\begin{enumerate}
\def\labelenumi{(\alph{enumi})}
\item
  Wire a little blue potentiometer dial to the Arduino
\item
  Copy and load the potentiometer.ion sketch onto the Arduino
\item
  Twist the potentiometer to turn the LED connected to the Arduino on
  and off
\item
  Use the Arduino Serial Monitor and Serial Plotter to see the
  potentiometer reading
\item
  Build a Python script and to record the potentiometer level and draw a
  plot
\end{enumerate}

    You will complete this lab in groups. Below are details for each of the
steps outlined above.

\hypertarget{part-1.-turn-an-led-on-and-off-with-python-and-an-arduino}{%
\subsubsection{Part 1. Turn an LED on and off with Python and an
Arduino}\label{part-1.-turn-an-led-on-and-off-with-python-and-an-arduino}}

    \hypertarget{a-download-the-arduino-ide}{%
\paragraph{(a) Download the Arduino
IDE}\label{a-download-the-arduino-ide}}

Download the Arduino IDE using the following link:

\begin{quote}
https://www.arduino.cc/en/Main/Software
\end{quote}

Scroll down the page to the Download the Arduino IDE section. Be sure to
select: \textbf{Windows ZIP for for non-admin install} as new software
can not be installed on lab computers without administrator privileges.
You can select \textbf{JUST DOWNLOAD} from the donation screen. Extract
the downloaded .zip folder to your thumb drive or the desktop.

    \hypertarget{b-wire-an-led-to-the-arduino}{%
\paragraph{(b) Wire an LED to the
Arduino}\label{b-wire-an-led-to-the-arduino}}

From the Arduino kit, take out an LED, a 330 Ohm resistor and two jumper
wires. Connect the LED, resistor and wires as shown below. Also see the
SIK GUIDE page 19 and the SparkFun Iventor's kit online guide:
https://learn.sparkfun.com/tutorials/sik-experiment-guide-for-arduino---v33/experiment-1-blinking-an-led
Note that LED's have two different length ``legs''. The short leg
connects to ground (thru a resistor), the long leg connects to Pin 13 on
the Arduino.

\begin{figure}
\centering
\includegraphics{images/Arduino_LED_fritzing.png}
\caption{Arduino with LED Frizting}
\end{figure}

    \hypertarget{c-upload-the-arduino-example-sketch-blink.ino-onto-the-arduino.-confirm-your-arduino-and-led-blinks.}{%
\paragraph{\texorpdfstring{(c) Upload the Arduino example sketch
\textbf{Blink.ino} onto the Arduino. Confirm your Arduino and LED
blinks.}{(c) Upload the Arduino example sketch Blink.ino onto the Arduino. Confirm your Arduino and LED blinks.}}\label{c-upload-the-arduino-example-sketch-blink.ino-onto-the-arduino.-confirm-your-arduino-and-led-blinks.}}

Open the Arduino IDE folder and open the Arduino.exe program. Open the
Arduino \textbf{Blink.ino} sketch by going to: {[}File{]}
--\textgreater{} {[}Examples{]} --\textgreater{} {[}Basics{]}
--\textgreater{} {[}01.Blink{]}

Connect the Arduino to the computer using the red USB cable. Note that
USB ports in monitors sometimes do not work correctly with Arduinos. Use
a USB port which is part of the computer. In the Arduino IDE Window that
contains the \textbf{Blink.ino} sketch, click the check mark to Verify,
then click the arrow to Upload.

\begin{figure}
\centering
\includegraphics{images/Check_to_Verify.png}
\caption{Check to Verify}
\end{figure}

\begin{figure}
\centering
\includegraphics{images/Arrow_to_Upload.png}
\caption{Arrow to Upload}
\end{figure}

Once the upload is complete, the red LED on the Arduino should blink on
and off.

If you don't see the Arduino's LED blinking, you need to do some trouble
shooting: * Check the COM Port under {[}Tools{]} --\textgreater{}
{[}Ports{]} * Check which type of board is selected under {[}Tools{]}
--\textgreater{} {[}Board{]}. \textbf{Arduino/Genuino Uno} needs to be
selected * Try unplugging and re-plugging in the Arduino's red USB
cable. Ensure the cable is seated in the computer and Arduino.

    \hypertarget{d-load-the-arduino-example-sketch-physicalpixel.ino}{%
\paragraph{(d) Load the Arduino example sketch
PhysicalPixel.ino}\label{d-load-the-arduino-example-sketch-physicalpixel.ino}}

Open the Arduino PhysicalPixel.ino sketch by going to: {[}File{]}
--\textgreater{} {[}Examples{]} --\textgreater{} {[}04.Communication{]}
--\textgreater{} {[}PhysicalPixel{]}. Once again, In the Arduino IDE
Window that contains the PhysicalPixel sketch, click the check mark to
Verify then click the arrow to Upload.

    \hypertarget{e-use-the-serial-monitor-to-turn-the-arduino-led-on-and-off}{%
\paragraph{(e) Use the serial monitor to turn the Arduino LED on and
off}\label{e-use-the-serial-monitor-to-turn-the-arduino-led-on-and-off}}

In the Arduino IDE Window that contains the PhysicalPixel sketch, open
the Serial Monitor by going to {[}Tools{]} --\textgreater{} {[}Serial
Monitor{]}.

In the Serial Monitor type: \texttt{H} and click {[}Send{]} (or press
ENTER). Then type: \texttt{L} and click {[}Send{]} (or press ENTER). You
should see the Arduino LED switch on and off. If the LED does not turn
on and off, make sure that the Port is set correctly in {[}Tools{]}
--\textgreater{} {[}Port{]} and make sure that the Serial Monitor is set
to 9600 baud.

    \hypertarget{f-use-the-python-repl-to-turn-the-arduino-led-on-and-off}{%
\paragraph{(f) Use the Python REPL to turn the Arduino LED on and
off}\label{f-use-the-python-repl-to-turn-the-arduino-led-on-and-off}}

At the Anaconda Prompt, type \texttt{\textgreater{}\ python} to enter
the Python REPL. At the Python REPL, type the following commands. If the
command is preceeded by a REPL prompt
\texttt{\textgreater{}\textgreater{}\textgreater{}} type the command
into the REPL. If the line does not start with a REPL prompt, this line
represents expected output.

\begin{verbatim}
>>> import serial
>>> import time

>>> ser = serial.Serial('COM4',9600)  # open serial port
>>> time.sleep(2)                      # wait 2 seconds for connection
>>> ser.name()
'COM4'

>>> ser.write(b'H')
# LED turns on

>>> ser.write(b'L')
# LED turns off

>>> ser.write(b'H')
# LED turns on

>>> ser.write(b'L')
# LED turns off

>>> ser.close()
>>> exit()
\end{verbatim}

    \hypertarget{f-build-a-python-script-to-turn-the-arduino-led-on-and-off}{%
\paragraph{(f) Build a Python script to turn the Arduino LED on and
off}\label{f-build-a-python-script-to-turn-the-arduino-led-on-and-off}}

If the Arduino is working correctly and the LED can be turned on and off
by typing \texttt{H} and \texttt{L} in the Serial Monitor, close the
Serial Monitor. If the Serial Monitor is left open, Python will not be
able to talk to the Arduino.

Now construct a Python script called \textbf{LED.py} which turns the
Arduino LED on and off, just like you were able to turn the LED on and
off with the Serial Monitor. To do this, you have to program Python to
send the characters \texttt{L} and \texttt{H} over the Serial line to
the Arduino.

    At the start of your Python script include a docstring
(\texttt{"""\ """}) section the contains a line with the program title,
and seperate lines that contain your name, the lab number and lab name,
course quarter and date. Below the docstring, start your script.

The first code section of your script needs to include imports for the
PySerial library. Include the line:

\begin{verbatim}
import serial
import time
\end{verbatim}

The next section of your script needs to set up the serial line for
communication. Note the Port = `COM4' line must be set according to the
port the Arduino is connected to. Insert the code below:

\begin{verbatim}
ser = serial.Serial('COM4')  # open serial port
print(ser.name)              # check which port was really used

# code to run          

ser.close() 
\end{verbatim}

Before you run the open serial port section of code, ensure put in place
the line to close the serial port. If the serial port is not closed, you
will not be able to use the serial port the next time your script runs.
When you have problems connecting to the Arduino with Python, often it
is because the serial line was not closed.

Insert one of the section of code below between the opening and closing
of the serial port:

\begin{verbatim}
# code to run 

# turns on LED
ser.write(b'H')
time.sleep(1)
\end{verbatim}

or

\begin{verbatim}
# code to run

# turns off LED
ser.write(b'L')
time.sleep(1)
\end{verbatim}

Run the entire script to ensure there are no errors and you can open and
close the serial port. A common error is the serial port COM\# is not
set correctly. Make sure you can get the LED to turn on and off by
running the two sections of code above.

Next, build a new section of code between the Open serial section and
Close serial sections. This section of code sends an H character over
the serial line waits a second, then sends a L character over the serial
line and waits another second. The code below is designed to blink the
Arduino LED on and off 10 times.

\begin{verbatim}
# code to run

for t in range(10);
    ser.write(b'H')
    time.sleep(1)
    ser.write(b'L')
    time.sleep(1)
\end{verbatim}

    Run the entire Python script and watch the Arduino LED blink 10 times. A
common problem is the serial port was not closed before the script
starts. Make sure the Arduino Serial Monitor is closed and try running
\texttt{\textgreater{}\textgreater{}\textgreater{}\ ser.close()} at the
Python REPL.

Once Python successfully blinks the Arduino LED, use your Python coding
skills to ask the user for input and turn on or off the Arduino's LED
based on the user input. Do this using the \texttt{input()} function
running in a while loop. Within the loop, add an if statement that will
allow the program to break out of the loop and go to the
\texttt{ser.close()} section. Remember the serial port must be closed
for Python to use the serial line again. Below is an example of an
if/break statement:

\begin{verbatim}
if user_input == 'q':
    break
    
\end{verbatim}

When your Python script runs, the user should see functionality like
below:

\begin{verbatim}
>>> Type H to turn on the LED, L to turn off the LED or q to quit: H
[LED TURNS ON]
>>> Type H to turn on the LED, L to turn off the LED or q to quit: L
[LED TURNS OFF]
>>> Type H to turn on the LED, L to turn off the LED or q to quit: q
[PROGRAM TERMINATES]
\end{verbatim}

    \hypertarget{part-2.-measure-sensor-output-with-python-and-an-arduino}{%
\subsubsection{Part 2. Measure sensor output with Python and an
Arduino}\label{part-2.-measure-sensor-output-with-python-and-an-arduino}}

Once you have completed the first two part of the lab and can turn the
Arduino LED on and OFF using Python, the next part of the lab is to use
read a sensor (a potentiometer dial) connected to the Arduino and plot
the data.

    \hypertarget{a-wire-a-potentiometer-dial-to-the-arduino}{%
\paragraph{(a) Wire a potentiometer dial to the
Arduino}\label{a-wire-a-potentiometer-dial-to-the-arduino}}

To do this, you first need to connect connect the potentiometer to the
Arduino. You should leave your LED and resistor attached as you will use
this with the potentiometer. Unplug the Arduino USB cable and connect
the small blue potentiometer dial to your Arduino as shown below. Also
see the SIK GUIDE page 25 and the SparkFun Iventor's kit online guide:
https://learn.sparkfun.com/tutorials/sik-experiment-guide-for-arduino---v33/experiment-2-reading-a-potentiometer

\begin{figure}
\centering
\includegraphics{images/redboard_pot_led_fritzing.png}
\caption{Arduino Potentiometer Fritzing Schematic}
\end{figure}

    \hypertarget{b-copy-and-load-the-potentiometer.ion-sketch-onto-the-arduino}{%
\paragraph{(b) Copy and load the potentiometer.ion sketch onto the
Arduino}\label{b-copy-and-load-the-potentiometer.ion-sketch-onto-the-arduino}}

Open a new sketch in the Arduino IDE by going to {[}File{]}
--\textgreater{} {[}New{]}. Copy the code below into the new sketch and
save the sketch as potentiometer \textbf{potentiometer.ino}

\begin{verbatim}
// potentiometer_read.ino
// reads a potentiometer and sends value over serial
int sensorPin = A0; // The potentiometer is connected to analog pin 0
int ledPin = 13; // The LED is connected to digital pin 13
int sensorValue; // an integer variable to store the potentiometer reading
void setup() // this function runs once when the sketch starts
{
// make the LED pin (pin 13) an output pin
pinMode(ledPin, OUTPUT);
// initialize serial communication:
Serial.begin(9600);
}
void loop() // this function runs repeatedly after setup() finishes
{
sensorValue = analogRead(sensorPin); // read the voltage at pin A0
Serial.println(sensorValue); // Output sensor value to Serial Monitor
if (sensorValue < 500) { // if sensor output is less than 500,
digitalWrite(ledPin, LOW); } // Turn the LED off
else { // if sensor output is greater than 500
digitalWrite(ledPin, HIGH); } // Keep the LED on
delay(100); // Pause 100 milliseconds before next reading
}
\end{verbatim}

Plug the Arduino USB cable back into the computer. In the Arduino IDE
window that contains the potentiometer \textbf{read.ino} sketch, click
the check mark to Verify the click the arrow to Upload the potentiometer
\textbf{read.ino} sketch to the Arduino.

    \hypertarget{c-twist-the-potentiometer-to-turn-the-led-connected-to-the-arduino-on-and-off}{%
\paragraph{(c) Twist the potentiometer to turn the LED connected to the
Arduino on and
off}\label{c-twist-the-potentiometer-to-turn-the-led-connected-to-the-arduino-on-and-off}}

After the potentiometer \textbf{potentiometer.ino} sketch is uploaded to
the Arduino, twist the small blue potentiometer dial back and forth and
watch the LED turn on and off. The on/off point should be about half way
through the little blue potentiometer's rotation. If the LED does not
turn on and off, double check your wiring and try uploading the sketch
again. Ensure the Serial Monitor is closed and that Python closed the
Serial Port before uploading.

    \hypertarget{d-use-the-arduino-serial-monitor-and-serial-plotter-to-see-the-potentiometer-reading}{%
\paragraph{(d) Use the Arduino Serial Monitor and Serial Plotter to see
the potentiometer
reading}\label{d-use-the-arduino-serial-monitor-and-serial-plotter-to-see-the-potentiometer-reading}}

Open the Arduino Serial Monitor in the Arduino IDE by going to
{[}Tools{]} --\textgreater{} {[}Serial Monitor{]}.

\begin{figure}
\centering
\includegraphics{images/Tools_SerialMonitor.png}
\caption{Arduino Tools --\textgreater{} Serial Monitor}
\end{figure}

You should see numbers scrolling down the Serial Monitor if the
potentiometer \textbf{read.ino} sketch is working properly. Twist the
little blue potentiometer and watch the numbers scrolling down the
Serial Monitor change.

\begin{figure}
\centering
\includegraphics{images/serial_monitor_output.png}
\caption{Arduino Serial Monitor}
\end{figure}

Next, close the Serial Monitor and open the Arduino Serial Plotter by
going to {[}Tools{]} --\textgreater{} {[}Serial Plotter{]}.

\begin{figure}
\centering
\includegraphics{images/Tools_SerialPlotter.png}
\caption{Arduino Tools --\textgreater{} Serial Plotter}
\end{figure}

You should see a plot with a moving line. Twist the little blue
potentiometer and observe the line on the plot move up and down. If the
Serial Plotter works, close the Serial Plotter Window. If Serial Plotter
isn't working, make sure the COM port is set correctly in the Arduino
IDE, and ensure the serial port was closed by Python. The Arduino Serial
Monitor and Serial Plotter can not be open at the same time and both
need to be closed before Python can communicate with the Arduino.

\begin{figure}
\centering
\includegraphics{images/serial_plotter_output.png}
\caption{Arduino Serial Plotter}
\end{figure}

    \hypertarget{e-use-the-python-repl-to-read-the-potentiometer-data}{%
\paragraph{(e) Use the Python REPL to read the potentiometer
data}\label{e-use-the-python-repl-to-read-the-potentiometer-data}}

At the Anaconda prompt, type \texttt{\textgreater{}\ python} to enter
the Python REPL. At the Python REPL, type the following commands. If the
command is preceeded by a REPL prompt
\texttt{\textgreater{}\textgreater{}\textgreater{}} type the command
into the REPL. If the line does not start with a REPL prompt, this line
represents expected output.

\begin{verbatim}
# serial read using the Python REPL

Type "help", "copyright", "credits" or "license" for more information.
>>> import serial
>>> import time
>>> ser = serial.Serial('COM4',9600)
>>> time.sleep(2)
>>> b = ser.readline()
>>> b
b'409\r\n'
>>> type(b)
<class 'bytes'>
>>> str_rn = b.decode()
>>> str_rn
'409\r\n'
>>> str = str_rn.rstrip()
>>> str
'409'
>>> type(str)
<class 'str'>
>>> f = float(str)
>>> f
409.0
>>> type(f)
<class 'float'>
>>> ser.close()
>>> exit()
\end{verbatim}

    \hypertarget{f-build-a-python-script-and-to-record-the-potentiometer-level-and-draw-a-plot}{%
\paragraph{(f) Build a Python script and to record the potentiometer
level and draw a
plot}\label{f-build-a-python-script-and-to-record-the-potentiometer-level-and-draw-a-plot}}

After you can successfully turn the Arduino LED on and off by twisting
the little blue potentiometer, and can see the plot line moving up and
down in the Serial Plotter, build a Python script which reads and plots
the potentiometer value.

To read the potentiometer value with Python, a new script called
\textbf{potentiometer.py} should be created. At the start of the Python
script include the standard docstring header within triple quotes
\texttt{"""\ """} Include a line for a title, and lines for your name,
the lab number and name, course/quarter and date.

The first lines of code in the script need to import the PySerial
package and the time module:

\begin{verbatim}
import serial
import time
\end{verbatim}

The next section of the script needs to set up the serial line for
communication and read the sensor value of the potentiometer. Note the
Port (\texttt{\textquotesingle{}COM4\textquotesingle{}}) must be set
according to the port the Arduino is connected to.

\begin{verbatim}
# set up the serial line
ser = serial.Serial('COM4', 9600)
time.sleep(2)
print(ser.name)

# Read and record the data
data =[]                           # initialize and empty list to store the data
for i in range(50):
    b = ser.readline()             # read a byte string line from the Arduino's serial output
    string_n = b.decode()          # decode byte string into regular string with \n and \r 
    string = string_n.rstrip()     # remove \n and \r from string
    flt = float(string)            # convert string to float
    print(flt)
    data.append(flt)               # add float to the end of the data list
    time.sleep(0.1)                # wait (sleep) 0.2 seconds before the next reading

ser.close()

# Plot the data
\end{verbatim}

Run the script and twist the potentiometer. You should see the
potentiometer values running by in the Python REPL command window.

Finally, fill in the remaining two sections in the script. Include a
section that asks a user for how long to collect data. Limit the user to
a maximum of 60 seconds. Then include a section that plots the
potentiometer reading vs.~time. Note that the data is recorded every
10th of a second.

When the Python script runs, it should function in the following way:

\begin{verbatim}
Enter the number of seconds to record data (0 - 60) : 10
\end{verbatim}

\begin{figure}
\centering
\includegraphics{images/potentiometer_reading.png}
\caption{matplotlib plot of potentiometer data}
\end{figure}

    \hypertarget{going-further}{%
\subsubsection{Going further}\label{going-further}}

If you have time, there is some additional functionality you can add to
your Python script:

\begin{itemize}
\item
  Display the high and low potentiometer readings
\item
  Connect a photocell to the Arduino and plot photocell readings with
  Python
\item
  Build a dynamic plot using an animated line that changes as the
  potentiometer value changes
\end{itemize}

    \hypertarget{deliverables}{%
\subsection{Deliverables}\label{deliverables}}

Make sure your group's Python and Arduino code are well commented and
sectioned. Ensure the variable names are descriptive and there is enough
documentation for another group of students to reuse the code without
much trouble. Each student's submission for the lab should be one of the
following files (you do not need to submit all the .py files used in the
lab, just the .py files you were primarily responsible for). Upload
these two files to the D2L Lab 8 Uploads folder.

\textbf{LED.py}

\textbf{potentiometer.py}

    \hypertarget{by-p.-kazarinoff-portland-community-college-2018}{%
\paragraph{\texorpdfstring{\emph{By P. Kazarinoff, Portland Community
College,
2018}}{By P. Kazarinoff, Portland Community College, 2018}}\label{by-p.-kazarinoff-portland-community-college-2018}}


    % Add a bibliography block to the postdoc
    
    
    
    \end{document}

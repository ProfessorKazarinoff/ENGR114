%% ENGR114_lab_assignment.tplx %%
   
\documentclass[11pt]{article}

 \input{preable.tex}   
 

 \begin{document}
    
    \hypertarget{lab-09---external-hardware}{%
\section{Lab 09 - External Hardware}\label{lab-09---external-hardware}}

    In this lab, you will use Python to communicate to a piece of external
hardware.

    \hypertarget{prelab}{%
\subsection{Prelab}\label{prelab}}

Before starting the lab, familiarize yourself with what an Arduino is
and what an Arduino can be used for. You will find numerous projects if
you Google ``Arduino project''. Download the Arduino IDE using the
following link:

\begin{quote}
https://www.arduino.cc/en/Main/Software
\end{quote}

Be sure to select: \textbf{{[}Windows ZIP for non-admin install{]}} as
new software can not be installed on lab computers without administrator
privileges.

\begin{figure}[H]
\centering
\includegraphics{images/arduino_download_page.png}
\caption{Arduino IDE Download Page}
\end{figure}

Investigate the Arudino IDE (launch by double clicking Arduino.exe). See
what is available in the {[}File{]} --\textgreater{} {[}Examples{]}
menu. Before lab, also familiarize yourself with the concept of serial
communication. In this lab, Python will communicate with an Arduino over
a serial connection. Serial communication is one of the older computer
hardware communication specifications. Serial communication is a
precursor to USB (universal \emph{serial} bus) communication used by
keyboards, mice, printers, thumb drives etc.

The first part of this lab will follow a tutorial from the Sparkfun
Inventor's Kit:

\begin{quote}
https://www.sparkfun.com/products/retired/12060
\end{quote}

Click on the \textbf{{[}Documents{]}} link on the product page to review
the guide.

    \hypertarget{lab}{%
\subsection{Lab}\label{lab}}

For this group lab assignment, your group will use Python to interact
with an Arduino to dynamically turn on and off an LED, and then use
Python to collect and plot the sensor data. Your group will construct
two Python scripts and use two Arduino scripts to complete these tasks.
At the end of the lab, you will be able to use Python to interact with
external hardware.

    An outline of the steps to complete the lab is below:

\hypertarget{part-1-turn-an-led-on-and-off-with-python-and-an-arduino}{%
\subsubsection{Part 1: Turn an LED on and off with Python and an
Arduino}\label{part-1-turn-an-led-on-and-off-with-python-and-an-arduino}}

\begin{enumerate}
\def\labelenumi{(\alph{enumi})}
\item
  Download the Arduino IDE
\item
  Wire an LED and resistor to the Arduino
\item
  Upload the Arduino example sketch \textbf{\emph{blink.ino}} onto the
  Arduino. Confirm your Arduino and LED blinks.
\item
  Load the Arduino example sketch \textbf{\emph{PhysicalPixel.ino}}
\item
  Use the serial monitor to turn the Arduino LED on and off
\item
  Build a Python script to turn the Arduino LED on and off
\end{enumerate}

    \hypertarget{part-2-measure-sensor-output-with-python-and-an-arduino}{%
\subsubsection{Part 2: Measure sensor output with Python and an
Arduino}\label{part-2-measure-sensor-output-with-python-and-an-arduino}}

\begin{enumerate}
\def\labelenumi{(\alph{enumi})}
\item
  Wire a little blue potentiometer dial to the Arduino
\item
  Copy and load the \textbf{\emph{potentiometer.ino}} sketch onto the
  Arduino
\item
  Twist the potentiometer to turn the LED connected to the Arduino on
  and off
\item
  Use the Arduino Serial Monitor and Arduino Serial Plotter to see the
  potentiometer reading
\item
  Build a Python script and to record the potentiometer level and draw a
  plot
\end{enumerate}

    You will complete this lab in groups. Below are details for each of the
steps outlined above.

\hypertarget{part-1.-turn-an-led-on-and-off-with-python-and-an-arduino}{%
\subsubsection{Part 1. Turn an LED on and off with Python and an
Arduino}\label{part-1.-turn-an-led-on-and-off-with-python-and-an-arduino}}

    \hypertarget{a-download-the-arduino-ide}{%
\paragraph{(a) Download the Arduino
IDE}\label{a-download-the-arduino-ide}}

Download the Arduino IDE using the following link:

\begin{quote}
https://www.arduino.cc/en/Main/Software
\end{quote}

Scroll down the page to the \textbf{{[}Download the Arduino IDE{]}}
section. Be sure to select: \textbf{{[}Windows ZIP for for non-admin
install{]}} as new software can not be installed on lab computers
without administrator privileges. You can select \textbf{{[}JUST
DOWNLOAD{]}} from the donation screen. Extract the downloaded .zip
folder to your thumb drive or the desktop.

    \hypertarget{b-wire-an-led-to-the-arduino}{%
\paragraph{(b) Wire an LED to the
Arduino}\label{b-wire-an-led-to-the-arduino}}

From the Arduino kit, take out an LED (any color), a 330 Ohm resistor
and two jumper wires. Connect the LED, resistor and wires as shown
below. Also see the SIK GUIDE page 19 and the SparkFun Iventor's kit
online guide:

\begin{quote}
https://learn.sparkfun.com/tutorials/sparkfun-inventors-kit-experiment-guide---v40/circuit-1a-blink-an-led
\end{quote}

Note that LED's have two different length ``legs''. The short leg
connects to ground (thru a resistor), the long leg connects to Pin 13 on
the Arduino.

\begin{figure}
\centering
\includegraphics{images/Arduino_LED_fritzing.png}
\caption{Arduino with LED Frizting}
\end{figure}

    \hypertarget{c-upload-the-arduino-example-sketch-blink.ino-onto-the-arduino.-confirm-your-arduino-and-led-blinks.}{%
\paragraph{\texorpdfstring{(c) Upload the Arduino example sketch
\textbf{Blink.ino} onto the Arduino. Confirm your Arduino and LED
blinks.}{(c) Upload the Arduino example sketch Blink.ino onto the Arduino. Confirm your Arduino and LED blinks.}}\label{c-upload-the-arduino-example-sketch-blink.ino-onto-the-arduino.-confirm-your-arduino-and-led-blinks.}}

Open the Arduino IDE folder and open the Arduino.exe program. Open the
Arduino \textbf{\emph{Blink.ino}} sketch by going to: {[}File{]}
--\textgreater{} {[}Examples{]} --\textgreater{} {[}Basics{]}
--\textgreater{} {[}01.Blink{]}.

Connect the Arduino to the computer using the red USB cable. Note that
USB ports in monitors sometimes do not work correctly with Arduinos. Use
a USB port which is part of the computer. In the Arduino IDE Window that
contains the \textbf{\emph{Blink.ino}} sketch, click the check mark to
Verify, then click the arrow to Upload.

\begin{figure}
\centering
\includegraphics{images/Check_to_Verify.png}
\caption{Check to Verify}
\end{figure}

\begin{figure}
\centering
\includegraphics{images/Arrow_to_Upload.png}
\caption{Arrow to Upload}
\end{figure}

Once the upload is complete, the red LED on the Arduino should blink on
and off.

If you don't see the Arduino's LED blinking, you need to do some trouble
shooting:

\begin{itemize}
\tightlist
\item
  Check the COM Port under {[}Tools{]} --\textgreater{} {[}Ports{]}
\item
  Check which type of board is selected under {[}Tools{]}
  --\textgreater{} {[}Board{]}. \textbf{Arduino/Genuino Uno} needs to be
  selected
\item
  Try unplugging and re-plugging in the Arduino's red USB cable. Ensure
  the cable is seated in the computer and Arduino.
\end{itemize}

    \hypertarget{d-load-the-arduino-example-sketch-physicalpixel.ino}{%
\paragraph{(d) Load the Arduino example sketch
PhysicalPixel.ino}\label{d-load-the-arduino-example-sketch-physicalpixel.ino}}

Open the Arduino \textbf{\emph{PhysicalPixel.ino}} sketch by going to:
{[}File{]} --\textgreater{} {[}Examples{]} --\textgreater{}
{[}04.Communication{]} --\textgreater{} {[}PhysicalPixel{]}. Once again,
In the Arduino IDE Window that contains the PhysicalPixel sketch, click
the check mark to Verify, then click the arrow to Upload.

    \hypertarget{e-use-the-serial-monitor-to-turn-the-arduino-led-on-and-off}{%
\paragraph{(e) Use the serial monitor to turn the Arduino LED on and
off}\label{e-use-the-serial-monitor-to-turn-the-arduino-led-on-and-off}}

In the Arduino IDE Window that contains the
\textbf{\emph{PhysicalPixel.ino}} sketch, open the Serial Monitor by
going to {[}Tools{]} --\textgreater{} {[}Serial Monitor{]}.

In the Serial Monitor type: \texttt{H} and click {[}Send{]} (or press
ENTER). Then type: \texttt{L} and click {[}Send{]} (or press ENTER). You
should see the Arduino LED switch on and off. If the LED does not turn
on and off, make sure that the Port is set correctly in {[}Tools{]}
--\textgreater{} {[}Port{]} and make sure that the Serial Monitor is set
to 9600 baud.

    \hypertarget{f-use-the-python-repl-to-turn-the-arduino-led-on-and-off}{%
\paragraph{(f) Use the Python REPL to turn the Arduino LED on and
off}\label{f-use-the-python-repl-to-turn-the-arduino-led-on-and-off}}

At the Anaconda Prompt, type \texttt{\textgreater{}\ python} to enter
the Python REPL. At the Python REPL, type the following commands. If the
command is preceeded by a REPL prompt
\texttt{\textgreater{}\textgreater{}\textgreater{}} type the command
into the REPL. If the line does not start with a REPL prompt, this line
represents expected output.

\begin{verbatim}
>>> import serial
>>> import time

>>> ser = serial.Serial('COM4',9600)  # open serial port
>>> time.sleep(2)                     # wait 2 seconds for connection
>>> ser.name()
'COM4'

>>> ser.write(b'H')
# LED turns on

>>> ser.write(b'L')
# LED turns off

>>> ser.write(b'H')
# LED turns on

>>> ser.write(b'L')
# LED turns off

>>> ser.close()
>>> exit()
\end{verbatim}

    \hypertarget{f-build-a-python-script-to-turn-the-arduino-led-on-and-off}{%
\paragraph{(f) Build a Python script to turn the Arduino LED on and
off}\label{f-build-a-python-script-to-turn-the-arduino-led-on-and-off}}

If the Arduino is working correctly and the LED can be turned on and off
by typing \texttt{H} and \texttt{L} in the Serial Monitor, close the
Serial Monitor. If the Serial Monitor is left open, Python will not be
able to talk to the Arduino.

Now construct a Python script called \textbf{LED.ipynb} which turns the
Arduino LED on and off, just like you were able to turn the LED on and
off with the Serial Monitor. To do this, you have to program Python to
send the characters \texttt{L} and \texttt{H} over the Serial line to
the Arduino.

    At the start of your Python script include a docstring
(\texttt{"""\ """}) section the contains a line with the program title,
and separate lines that contain your name, the lab number and lab name,
course quarter and date. Below the docstring, start your script.

The first code section of your script needs to include imports for the
PySerial library. Include the lines:

\begin{verbatim}
import serial
import time
\end{verbatim}

The next section of your script needs to set up the serial line for
communication. Note the
\texttt{Port\ =\ \textquotesingle{}COM4\textquotesingle{}} line must be
set according to the port the Arduino is connected to. Insert the code
below:

\begin{verbatim}
ser = serial.Serial('COM4')  # open serial port
print(ser.name)              # check which port was really used

# code to run          

ser.close() 
\end{verbatim}

Before you run the open serial port section of code, put in place the
line of code to close the serial port. If the serial port is not closed,
you will not be able to use the serial port the next time your script
runs. When you have problems connecting to the Arduino with Python,
often it is because the serial line was not closed.

Insert one of the sections of code below between the opening and closing
of the serial port:

\begin{verbatim}
# turn on LED
ser.write(b'H')
time.sleep(1)
\end{verbatim}

or

\begin{verbatim}
# turn off LED
ser.write(b'L')
time.sleep(1)
\end{verbatim}

Run the entire script to ensure there are no errors and you can open and
close the serial port. A common error is the serial port
\texttt{\textquotesingle{}COM\#\textquotesingle{}} is not set correctly.
Make sure you can turn the LED turns on and off by running the two sections
of code above.

Next, build a new section of code between the Open serial section and
Close serial section. This section of code sends an
\texttt{\textquotesingle{}H\textquotesingle{}} character over the serial
line waits a second, then sends a
\texttt{\textquotesingle{}L\textquotesingle{}} character over the serial
line and waits another second. The code below is designed to blink the
LED on and off 10 times.

\begin{verbatim}
# code to run

for t in range(10);
    ser.write(b'H')
    time.sleep(1)
    ser.write(b'L')
    time.sleep(1)
\end{verbatim}

    Run the entire Python script and watch the Arduino LED blink 10 times. A
common problem is the serial port was not closed before the script
starts. Make sure the Arduino Serial Monitor is closed and try running
\texttt{\textgreater{}\textgreater{}\textgreater{}\ ser.close()} at the
Python REPL.

After you successfully blink the LED with Python, use your Python coding
skills to ask a user for input and turn on or off the Arduino's LED
based on user input. Complete this task with the \texttt{input()}
function running in a while loop. Within the while loop, add an if-statement that allows the program to break out of the loop (remember the
keyword \texttt{break}) and go to the \texttt{ser.close()} section. The
serial port must be closed for Python to use the serial line again.

Below is an example of an if/break statement:

\begin{verbatim}
if user_input == 'q':
    break
    
\end{verbatim}

When your Python script runs, the user should see functionality like
below:

\begin{verbatim}
>>> Type H to turn on the LED, L to turn off the LED or q to quit: H
[LED TURNS ON]
>>> Type H to turn on the LED, L to turn off the LED or q to quit: L
[LED TURNS OFF]
>>> Type H to turn on the LED, L to turn off the LED or q to quit: q
[PROGRAM TERMINATES]
\end{verbatim}

\newpage

    \hypertarget{part-2.-measure-sensor-output-with-python-and-an-arduino}{%
\subsubsection{Part 2. Measure sensor output with Python and an
Arduino}\label{part-2.-measure-sensor-output-with-python-and-an-arduino}}

Once you have completed Part 1 of the lab and can turn the Arduino LED
on and OFF using Python continue on to Part 2. In Part 2 of the lab, you
will use Python to read a sensor (a potentiometer dial) connected to the
Arduino and plot the data.

    \hypertarget{a-wire-a-potentiometer-dial-to-the-arduino}{%
\paragraph{(a) Wire a potentiometer dial to the
Arduino}\label{a-wire-a-potentiometer-dial-to-the-arduino}}

First, you need to connect connect the potentiometer to the Arduino. You
should leave your LED and resistor attached as you will use the LED and
resistor with the potentiometer. Unplug the Arduino USB cable and
connect the small blue potentiometer dial to your Arduino as shown
below. Also see the SIK GUIDE page 25 and the SparkFun Iventor's kit
online guide:

\begin{quote}
https://learn.sparkfun.com/tutorials/sparkfun-inventors-kit-experiment-guide---v40/circuit-1b-potentiometer

\end{quote}

\begin{figure}[H]
\centering
\includegraphics{images/redboard_pot_led_fritzing.png}
\caption{Arduino Potentiometer Fritzing Schematic}
\end{figure}

\newpage

    \hypertarget{b-copy-and-load-the-potentiometer.ino-sketch-onto-the-arduino}{%
\paragraph{(b) Copy and load the potentiometer.ino sketch onto the
Arduino}\label{b-copy-and-load-the-potentiometer.ino-sketch-onto-the-arduino}}

Open a new sketch in the Arduino IDE by going to {[}File{]}
--\textgreater{} {[}New{]}. Copy the code below into the new sketch and
save the sketch as \textbf{\emph{potentiometer.ino}}

\begin{verbatim}
// potentiometer_read.ino
// reads a potentiometer and sends value over serial

int sensorPin = A0; // The potentiometer is connected to analog pin 0
int ledPin = 13; // The LED is connected to digital pin 13
int sensorValue; // an integer variable to store the potentiometer reading

void setup() // this function runs once when the sketch starts
{
// make the LED pin (pin 13) an output pin
pinMode(ledPin, OUTPUT);
// initialize serial communication:
Serial.begin(9600);
}

void loop() // this function runs repeatedly after setup() finishes
{
sensorValue = analogRead(sensorPin); // read the voltage at pin A0
Serial.println(sensorValue); // Output sensor value to Serial Monitor
if (sensorValue < 500) { // if sensor output is less than 500,
digitalWrite(ledPin, LOW); } // Turn the LED off
else { // if sensor output is greater than 500
digitalWrite(ledPin, HIGH); } // Keep the LED on
delay(100); // Pause 100 milliseconds before next reading
}
\end{verbatim}

Plug the Arduino USB cable back into the computer. In the Arduino IDE
window that contains the potentiometer \textbf{\emph{potentiometer.ino}}
sketch, click the check mark to Verify the click the arrow to Upload the
potentiometer \textbf{\emph{potentiometer.ino}} sketch to the Arduino.

    \hypertarget{c-twist-the-potentiometer-to-turn-the-led-connected-to-the-arduino-on-and-off}{%
\paragraph{(c) Twist the potentiometer to turn the LED connected to the
Arduino on and
off}\label{c-twist-the-potentiometer-to-turn-the-led-connected-to-the-arduino-on-and-off}}

After the potentiometer \textbf{\emph{potentiometer.ino}} sketch is
uploaded to the Arduino, twist the small blue potentiometer dial back
and forth and watch the LED turn on and off (the dial can be stiff). The
on/off point should be about half way through the little blue
potentiometer's rotation. If the LED does not turn on and off, double
check your wiring and try uploading the
\textbf{\emph{potentiometer.ino}} sketch again. Ensure the Serial
Monitor is closed and that Python closed the Serial Port before you
upload the sketch.

    \hypertarget{d-use-the-arduino-serial-monitor-and-serial-plotter-to-see-the-potentiometer-reading}{%
\paragraph{(d) Use the Arduino Serial Monitor and Serial Plotter to see
the potentiometer
reading}\label{d-use-the-arduino-serial-monitor-and-serial-plotter-to-see-the-potentiometer-reading}}

Open the Arduino Serial Monitor in the Arduino IDE by going to
{[}Tools{]} --\textgreater{} {[}Serial Monitor{]}.

\begin{figure}
\centering
\includegraphics{images/Tools_SerialMonitor.png}
\caption{Arduino Tools --\textgreater{} Serial Monitor}
\end{figure}

You should see numbers scrolling down the Serial Monitor if the
potentiometer \textbf{\emph{potentiometer.ino}} sketch is working
properly. Twist the little blue potentiometer and watch the numbers
scrolling down the Serial Monitor change.

\begin{figure}
\centering
\includegraphics{images/serial_monitor_output.png}
\caption{Arduino Serial Monitor}
\end{figure}

Next, close the Arduino Serial Monitor and open the Arduino Serial
Plotter by going to {[}Tools{]} --\textgreater{} {[}Serial Plotter{]}.

\begin{figure}
\centering
\includegraphics{images/Tools_SerialPlotter.png}
\caption{Arduino Tools --\textgreater{} Serial Plotter}
\end{figure}

You should see a plot with a moving line. Twist the little blue
potentiometer and observe the line on the plot move up and down.

If the Serial Plotter works, close the Serial Plotter Window. If Serial
Plotter isn't working, make sure the
\texttt{\textquotesingle{}COM\textquotesingle{}} port is set correctly
in the Arduino IDE, and ensure the serial port was closed by Python. The
Arduino Serial Monitor and Serial Plotter can not be open at the same
time, and both need to be closed before Python can communicate with the
Arduino.

\begin{figure}[H]
\centering
\includegraphics{images/serial_plotter_output.png}
\caption{Arduino Serial Plotter}
\end{figure}

    \hypertarget{e-use-the-python-repl-to-read-the-potentiometer-data}{%
\paragraph{(e) Use the Python REPL to read the potentiometer
data}\label{e-use-the-python-repl-to-read-the-potentiometer-data}}

At the Anaconda prompt, type \texttt{\textgreater{}\ python} to enter
the Python REPL. At the Python REPL, type the following commands. If the
command is preceeded by a REPL prompt
\texttt{\textgreater{}\textgreater{}\textgreater{}} type the command
into the REPL. If the line does not start with a REPL prompt, this line
represents expected output.

\begin{verbatim}
# serial read using the Python REPL

>>> import serial
>>> import time
>>> ser = serial.Serial('COM4',9600)
>>> time.sleep(2)
>>> b = ser.readline()
>>> b
b'409\r\n'
>>> type(b)
<class 'bytes'>
>>> str_rn = b.decode()
>>> str_rn
'409\r\n'
>>> str = str_rn.rstrip()
>>> str
'409'
>>> type(str)
<class 'str'>
>>> f = float(str)
>>> f
409.0
>>> type(f)
<class 'float'>
>>> ser.close()
>>> exit()
\end{verbatim}

    \hypertarget{f-build-a-python-script-and-to-record-the-potentiometer-level-and-draw-a-plot}{%
\paragraph{(f) Build a Python script and to record the potentiometer
level and draw a
plot}\label{f-build-a-python-script-and-to-record-the-potentiometer-level-and-draw-a-plot}}

After you can successfully turn the Arduino LED on and off by twisting
the little blue potentiometer, and can see the plot line moving up and
down in the Serial Plotter, build a Python script which reads and plots
the potentiometer value.

To read the potentiometer value with Python, a new script called
\textbf{potentiometer.ipynb} should be created. At the start of the
Python script include the standard docstring header within triple quotes
\texttt{"""\ """} Include a line for a title, and lines for your name,
the lab number and name, course/quarter and date.

The first lines of code in the script need to import the
\textbf{PySerial} package and the time module:

\begin{verbatim}
import serial
import time
\end{verbatim}

The next section of the script needs to set up the serial line for
communication and read the sensor value of the potentiometer. Note the
Port (\texttt{\textquotesingle{}COM4\textquotesingle{}}) must be set
according to the port the Arduino is connected to.

\begin{verbatim}
# set up the serial line
ser = serial.Serial('COM4', 9600)
time.sleep(2)
print(ser.name)

# Read and record the data
data =[]                           # initialize and empty list to store the data
for i in range(50):
    b = ser.readline()             # read a byte string line from the Arduino's serial output
    string_n = b.decode()          # decode byte string into regular string with \n and \r 
    string = string_n.rstrip()     # remove \n and \r from string
    flt = float(string)            # convert string to float
    print(flt)
    data.append(flt)               # add float to the end of the data list
    time.sleep(0.1)                # wait (sleep) 0.2 seconds before the next reading

ser.close()

# Plot the data
\end{verbatim}

Run the script and twist the potentiometer. You should see the
potentiometer values running by in the Python REPL command window.

Finally, fill in the remaining two sections in the script. Include a
section that asks a user for how long to collect data. Limit the user to
a maximum of 60 seconds. Then include a section that plots the
potentiometer reading vs.~time. Note that the data is recorded every
10th of a second.

When the Python script runs, it should function in the following way:

\begin{verbatim}
Enter the number of seconds to record data (0 - 60) : 10
\end{verbatim}

\begin{figure}[H]
\centering
\includegraphics{images/potentiometer_reading.png}
\caption{matplotlib plot of potentiometer data}
\end{figure}

    \hypertarget{going-further}{%
\subsubsection{Going further}\label{going-further}}

If you have time, there is some additional functionality you can add to
your Python script:

\begin{itemize}
\item
  Display the high and low potentiometer readings
\item
  Connect a photocell to the Arduino and plot photocell readings with
  Python
\item
  Build a dynamic plot using an animated line that changes as the
  potentiometer value changes
\end{itemize}

    \hypertarget{deliverables}{%
\subsection{Deliverables}\label{deliverables}}

Make sure your group's Python and Arduino code are well commented and
sectioned. Ensure the variable names are descriptive and there is enough
documentation for another group of students to reuse the code without
much trouble. Each student's submission for the lab should be one of the
following files (you do not need to submit all the .ipynb files used in
the lab, just the files you were primarily responsible for). Each
student needs to upload one Python file and one Arduino file. Upload
your files to the D2L Lab 8 Uploads folder.

\textbf{LED.ipynb} and \textbf{PhysicalPixel.ino}

or

\textbf{potentiometer.ipynb} and \textbf{potentiometer.ino}

    \hypertarget{by-p.-kazarinoff-portland-community-college-2018}{%
\paragraph{\texorpdfstring{\emph{By P. Kazarinoff, Portland Community
College,
2018}}{By P. Kazarinoff, Portland Community College, 2018}}\label{by-p.-kazarinoff-portland-community-college-2018}}

\end{document}

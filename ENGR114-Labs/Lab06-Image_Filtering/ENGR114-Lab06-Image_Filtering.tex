%% ENGR114_lab_assignment.tplx %%
%
% Built off of the article.tplx template %

% Default to the notebook output style
% Inherit from the specified cell style.

\documentclass[11pt]{article}

\input{preamble.tex}
    
\begin{document}
  
\hypertarget{lab-06---image-filtering}{%
\section{Lab 06 - Image Filtering}\label{lab-06---image-filtering}}

\hypertarget{prelab}{%
\subsection{Prelab}\label{prelab}}

Before starting the lab, read through this entire document. Then review
how Matplotlib can be used to display images. The Matplotlib
documentaiton on showing images with \texttt{plt.imshow()} is found
here:
\url{https://matplotlib.org/api/_as_gen/matplotlib.pyplot.imshow.html}.
Also be familiar with Python's \texttt{input()} function and \texttt{if}
/ \texttt{elif} / \texttt{else} selection structures.

\hypertarget{lab}{%
\subsection{Lab}\label{lab}}

In this lab assignment, you will show a user an image and solicit input
from the user regarding how to modify the image. Based on the user's
input, you will re-display the image with modifications.

The image below is the image the user will modify:

\begin{figure}[!h]
\centering
\includegraphics[scale=0.3]{stinkbug.png}
\caption{stinkbug}
\end{figure}

The image can be found on-line here:
\url{https://matplotlib.org/users/image_tutorial.html}

Images can be displayed using Python and Matplotlib. The code cell below
demonstrates how to read and display an image with Matplotlib. Ensure
the image you want to display is in the same directory as your Jupyter
notebook.

\begin{Verbatim}[commandchars=\\\{\}]
{\color{incolor}In [{\color{incolor}1}]:} \PY{k+kn}{import} \PY{n+nn}{numpy} \PY{k}{as} \PY{n+nn}{np}
        \PY{k+kn}{import} \PY{n+nn}{matplotlib}\PY{n+nn}{.}\PY{n+nn}{pyplot} \PY{k}{as} \PY{n+nn}{plt}
        \PY{o}{\PYZpc{}}\PY{k}{matplotlib} inline
        
        \PY{n}{image} \PY{o}{=} \PY{n}{plt}\PY{o}{.}\PY{n}{imread}\PY{p}{(}\PY{l+s+s1}{\PYZsq{}}\PY{l+s+s1}{stinkbug.png}\PY{l+s+s1}{\PYZsq{}}\PY{p}{)}
        
        \PY{n}{fig}\PY{p}{,} \PY{n}{ax} \PY{o}{=} \PY{n}{plt}\PY{o}{.}\PY{n}{subplots}\PY{p}{(}\PY{p}{)}
        \PY{n}{ax}\PY{o}{.}\PY{n}{imshow}\PY{p}{(}\PY{n}{image}\PY{p}{)}
        
        \PY{n}{plt}\PY{o}{.}\PY{n}{show}\PY{p}{(}\PY{p}{)}
\end{Verbatim}

\begin{figure}[!h]
\begin{center}
    \includegraphics[scale=0.5]{combined_files/combined_3_0.png}
\end{center}
\end{figure}
    
The \texttt{stinkbug.png} image is a gray scale image, but Matplotlib
renders it as a color image. We can include an extra argument in the
\texttt{ax.imshow()} method to choose a different color map. A list of
all the possible color maps can be found in the Matplotlib documentation
(\url{https://matplotlib.org/examples/color/colormaps_reference.html}).



The code cell below demonstartes how to use a custom color map to
display the image

\begin{Verbatim}[commandchars=\\\{\}]
{\color{incolor}In [{\color{incolor}2}]:} \PY{k+kn}{import} \PY{n+nn}{numpy} \PY{k}{as} \PY{n+nn}{np}
        \PY{k+kn}{import} \PY{n+nn}{matplotlib}\PY{n+nn}{.}\PY{n+nn}{pyplot} \PY{k}{as} \PY{n+nn}{plt}
        
        \PY{n}{image} \PY{o}{=} \PY{n}{plt}\PY{o}{.}\PY{n}{imread}\PY{p}{(}\PY{l+s+s1}{\PYZsq{}}\PY{l+s+s1}{stinkbug.png}\PY{l+s+s1}{\PYZsq{}}\PY{p}{)}
        
        \PY{n}{fig}\PY{p}{,} \PY{n}{ax} \PY{o}{=} \PY{n}{plt}\PY{o}{.}\PY{n}{subplots}\PY{p}{(}\PY{n}{figsize}\PY{o}{=}\PY{p}{(}\PY{l+m+mi}{9}\PY{p}{,}\PY{l+m+mi}{9}\PY{p}{)}\PY{p}{)}
        \PY{n}{ax}\PY{o}{.}\PY{n}{imshow}\PY{p}{(}\PY{n}{image}\PY{p}{,} \PY{n}{cmap}\PY{o}{=}\PY{l+s+s1}{\PYZsq{}}\PY{l+s+s1}{Greys}\PY{l+s+s1}{\PYZsq{}}\PY{p}{)}
        
        \PY{n}{plt}\PY{o}{.}\PY{n}{show}\PY{p}{(}\PY{p}{)}
\end{Verbatim}

\begin{figure}[!h]
\begin{center}
\includegraphics[scale=0.35]{combined_files/combined_5_0.png}
\end{center}
\end{figure}
    
Construct a script that tells a user the possible colormap options. Then
use Python's \texttt{input()} function to ask the user for a color map.
Next, ask the user if they want gridlines. Finally ask the user if they
want tick labels. After the user selects all of the options, show the
user the image based on their preferences.

\newpage

The code below demonstrates how the script might function:

\begin{verbatim}
Your color maps options are: Greys, gist_earth, ocean and terrain

Please select a color map:
terrain

Do you want grid lines? (yes or no):
yes

Do you want tick labels? (yes or no):
no
\end{verbatim}

\begin{figure}[!h]
\centering
\includegraphics[scale=0.4]{images/image1.png}
\caption{Image terrain colormap, no gridlines, no tick labels}
\end{figure}

\hypertarget{deliverables}{%
\subsection{Deliverables}\label{deliverables}}

Each student's submission for the lab should be one Jupyter notebook
.ipynb-file. Run your Jupyter notebook as a user and select
\texttt{terrain} as the colormap, \texttt{yes} gridlines and \texttt{no}
ticklabels. Show the resulting image. Upload to the appropriate D2L drop
box.

\textbf{lab6.ipynb}

\hypertarget{by-p.-kazarinoff-portland-community-college-2019}{%
\paragraph{\texorpdfstring{\emph{By P. Kazarinoff, Portland Community
College,
2019}}{By P. Kazarinoff, Portland Community College, 2019}}\label{by-p.-kazarinoff-portland-community-college-2019}}

\end{document}
